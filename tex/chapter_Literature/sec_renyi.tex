% !TeX root=../main.tex
\section{Аналіз існуючих підходів}

У 1958 році угорський математик Альфред Реньї опублікував працю щодо проблеми заповнення одновимірного обмеженого простору. Ця праця стала підґрунтям для подальшого дослідження в напрямі проблеми парковки та пакування \cite{MathWorldRenyi}.

Задача формулювалася наступним чином: нехай є заданий відрізок $[0;~x]$, де $x>1$, і нехай на цей відрізок "паркуються" одновимірні "автомобілі" одиничної довжини, керуючись рівномірним розподілом. У такій ситуації доводилось, що середнє максимальне значення кількості машин на парковці буде $M(x)$ задовільняє наступну систему:
\begin{equation}
M(X)=
\begin{cases}
0 &$0 \leq X < 1$ \\
1 + \frac{2}{X-1} \int\limits_0^{X-1} M(y) dy &$X \geq 1$.
\end{cases}
\end{equation}

В такому випадку середня щільність автомобілів для великих $X$ виходить
\begin{equation}
m = \lim\limits_{X \rightarrow \infty} \frac{M(X)}{X} = \int\limits_0^\infty \exp\left(-2 \int\limits_0^x \frac{1-e^{-y}}{y} dy \right) dx \approx 0.747597.
\end{equation}

Але цей результат потребує узагальнення, адже описана вище класична модель Реньї не дає можливості активно застосовувати результат на практиці. Тому у ції роботі буде створена більш загальна модель, в якій водії мають змішану модель поведінки: з деякою ймовірністю вони ставлять свій автомобіль впритул до сусіднього, а в інакшому випадку – керуючись рівномірним розподілом.

\todo{Fix this}
У роботах Дворецького \cite{Dvoretzky} та Блезделя \cite{Blaisdell} проведено аналогічне дослідження, але отримано більш точну асимптотичну оцінку:
\begin{equation}
M(X) = C X - (1 - C) + O\left(\frac{1}{|t|^n}\right).
\end{equation}

У роботі \cite{coffman1998parking} виконано аналіз моделі з пуассонівським розподілом у місці та часі.

У роботі \cite{exhaustion2017mackey} розглянуто модель паркування з трохи іншої строни --- розглядається порядок "виснаження" парковки у випадку заповнення з бінарним діленням довжини автомобіля, тобто швидкість росту зайнятого простору після заповнення проміжку автомобілями довжини 2, 1, $\frac{1}{2}, \dots$.



