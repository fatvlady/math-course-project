% !TeX root=../main.tex
\section{Аналіз існуючих підходів}

У 1958 році угорський математик Альфред Реньї опублікував працю щодо проблеми заповнення одновимірного обмеженого простору. Ця праця стала підґрунтям для подальшого дослідження в напрямі проблеми парковки та пакування \cite{MathWorldRenyi}.

Задача формулювалася наступним чином: нехай є заданий відрізок $[0;~x]$, де $x>1$, і нехай на цей відрізок "паркуються" одновимірні "автомобілі" одиничної довжини, керуючись рівномірним розподілом. У такій ситуації доводилось, що середнє максимальне значення кількості машин на парковці буде $m(x)$ задовільняє наступну систему:
\begin{equation}
m(X)=
\begin{cases}
0 &$0 \leq X < 1$ \\
1 + \frac{2}{X-1} \int\limits_0^{X-1} m(y) dy &$X \geq 1$.
\end{cases}
\end{equation}

В такому випадку середня щільність автомобілів для великих $X$ виходить
\begin{equation}
m = \lim\limits_{x \rightarrow \infty} \frac{m(x)}{x} = \int\limits_0^\infty \exp\left(-2 \int\limits_0^x \frac{1-e^{-y}}{y} dy \right) dx \approx 0.747597.
\end{equation}

Але цей результат потребує узагальнення, адже описана вище класична модель Реньї не дає можливості активно застосовувати результат на практиці. Тому у ції роботі буде створена більш загальна модель, в якій водії мають змішану модель поведінки: з деякою ймовірністю вони ставлять свій автомобіль впритул до сусіднього, а в інакшому випадку – керуючись рівномірним розподілом.

У роботі Дворецького \cite{Dvoretzky} проведено аналогічне дослідження, але отримано більш точну асимптотичну оцінку:
\begin{equation}
\label{eq:uniform_model_asymptotics_fine}
m(X) = C x - (1 - C) + \mathcal{O}\left(\left(\frac{2e}{x}\right)^{x-1.5}\right).
\end{equation}

Після отримання асимптотики для першого моменту кількості автомобілів на парковці \eqref{eq:uniform_model_asymptotics_fine} було показано асимптотичну нормальність величини
\begin{equation}
Z_{x} = \frac{N_{x} - m(x)}{\sigma(x)} \sim \operatorname{N}(0,1), \quad x \rightarrow \infty,
\end{equation}

де $N_{x}$ -- випадкова величина, що означає кількість автомобілів на парковці в термінальному стані, а $\sigma(x)$ -- стандартне відхилення $N_{x}$.

У роботі Блезделя \cite{blaisdell1970} описано чисельне вирахування константи $C$ у одновимірному та двовимірному випадках, тобто автомобілі вже квадрати зі стороною $1$, і паркуються на великий квадрат зі стороною $x$. Наводиться гіпотеза та чисельне підкріплення, що константа у двовимірному випадку дорівнює $C^2$. У 1974 році Лал \cite{lal1970} порахував значення константи $C$ з точністю до 19 знаку після коми.

У праці Нея \cite{ney1962} проведено аналіз широкого спектру моделей на базі класичної моделі Реньї. Він розглядав процедуру паркування з випадковою довжиною автомобіля, та показав, що при справджені певних умов регуляризації має сенс досліджувати моменти величини $N(b,x)$, що позначає кількість автомобілів довжини не менше аніж $b$ в термінальному стані на парковці довжини $x$. Було доведено, що з виконання наведених у роботі умов регуляризації середнє значення $N(b,x)$ прямує до лінійної функції від $x$ з порядком асимптотики $\mathcal{O}(x^{-n})$ для будь-якого цілого $n > 0$. У випадку рівномірного вибору місця паркування виведено лінійну залежність у явній формі.

Деякі досліджувачі підійшли до проблематики з іншої сторони. Наприклад, у роботі \cite{coffman1998parking} Коффман та ін. виконали аналіз моделі з он-лайн паркуванням за Пуассонівським процесом як для вибору місця прибуття, так і для моменту прибуття. Таким чином, починаючи с часу $0$ одиничні інтервали прибувають на $\RR^+$ з імовірністю $\Delta{t}\Delta{y} + \mathcal{o}(\Delta{t}\Delta{y})$ потрапити на проміжок $[y,~y+\Delta y]$ між часом $t$ та $t + \Delta t$. При фіксованій довжині парковки $x > 0$, інтервал приймається лише у тому випадку, якщо він потрапляє на $[0,~x]$ та не перекривається з існуючими на цей час інтервалами. Далі досліджується кількість $N_{x}(t)$ інтервалів, прийнятих до моменту $t$ включно. Для такої моделі було проведено виведення, дещо аналогічне виведенню для класичної моделі Реньї у праці \cite{Dvoretzky}, і було показано, що при $x \rightarrow \infty$, $\EE N_{x}(t) \rightarrow \alpha(t)x$ і $\DD N_{x}(t) \rightarrow \mu(t)x$ рівномірно на $t \in (0, T)$ для деякого фіксованого $T$, де $\alpha(t)$ та $\mu(t)$ задані явними, хоча і достатньо складними формулами. Використовуючи цю асимптотику, було доведено асимптотичну нормальність $N_{x}(t)$ для фіксованого $t$:
\begin{equation}
\frac{N_{x} - m(x)}{\sigma(x)} \xrightarrow{\operatorname{d}} \operatorname{N}(0,1), \quad x \rightarrow \infty.
\end{equation}

У наступній своїй праці \cite{coffman2000parking} Коффман та ін. досліджували ту саму модель з Пуассонівським процесом, але розглядали частину вільних проміжків довжини не більше $y$. Якщо позначити через $p(t,y)$ границю при $x \rightarrow \infty$ частини вільних проміжків на момент $t$, то було доведено наступний результат:
\begin{equation}
p(t,y)=\begin{cases}
\frac{2\int\limits_{0}{t} (1 - e^{-vy})\beta(v)\,dv}{\alpha(t)}, &\quad y \leq 1,\\
p(t,1) + \frac{(1 - e^{-t(y-1)})t\beta(t)}{\alpha(t)}, &\quad y > 1,
\end{cases}
\end{equation}

де $\alpha(t) = \int\limits_{0}{t}\beta(v)\,dv$ і $\beta(t) = \exp\left(-2\int\limits_{0}{t} \frac{1 - e^{-v}}{v}\,dv\right)$.

У наступних роботах розглянуто моделі, що заповнюють інтервал повністю. Наприклад, у праці Юлія Баришникова та Олександра Гнєдіна \cite{baryshnikov1962} було розглянуто модель заповнення інтервалу $[0,~1]$ інтервалами зі змінною довжиною за наступним законом:
\begin{equation}
\Prob{I \subset [x,~ 1 - y]} = (1 - x - y)^\alpha, \quad (x,y) \in \Delta,
\end{equation}

де $\alpha > 1$ та $\Delta =\{(x,y)| x \geq 0, y \geq 0, x + y \leq 1 \}$. Розглядається асимптотика кількості інтервалів, що були прийняті після $n$ спроб.

У більш сучасній роботі \cite{exhaustion2017mackey} парковка заповнюється інтервалами довжини 2, після виснаження, вільні проміжки заповнюються одиничними інтервалами, на наступній ітерації інтервалами довжини $\frac{1}{2}$, і т.д. Розглядається швидкість виснаження парковки, а саме величина $\frac{L_{n+1}}{L_{n}}$, де $L_{n}$ -- очікувана сумарна довжина вільних проміжків після $n$ ітерацій. Доводиться, що
\begin{equation}
\lim\limits_{n \rightarrow \infty} \frac{L_{n+1}}{L_{n}} = R_{\frac{1}{2}} \approx 0.61.
\end{equation}

Частину досліджень присвячено дискретним моделям. Наприклад, у роботі \cite{fleurke2009} було розглянуто парування у два ряди цілочисельної сітки $\mathbb{Z}$, а в праці \cite{dehling2008} автомобілі паркуються на вершини випадкового дерева. Робота Пояркова присвячена визначенню нижньої межі кількості розміщених кубів для процесу паркування $d$-розмірних кубів зі стороною 2 в $d$-розмірний куб зі стороною 4 до моменту сатурації. При чому куби з ребром довжини 2 паралельні сторонам великого кубу, і центри знаходяться в цілих точках. Доводиться, що середня кількість малих кубів не менша за $\left(\frac{3}{2}\right)^d$.

У роботі Пенроуза \cite{penrose2002} розглядається процес пакування одиничних куль в великий куб, аналогічно моделі Реньї, але з будь-якою розмірністю. Доводиться ЗВЧ та ЦГТ для кількості розміщених куль в термодинамічній границі.

%% WOOHOO CHEATING :)
\nocite{itoh1999}
\nocite{penrose2001cmp}
\nocite{itoh1986}
\nocite{solomon1986}
\nocite{mackenzie1962}
