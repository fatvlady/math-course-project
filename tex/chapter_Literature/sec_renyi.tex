% !TeX root=../main.tex
\section{Аналіз існуючих підходів}

У 1958 році угорський математик Альфред Реньї опублікував працю щодо проблеми заповнення одновимірного обмеженого простору. Ця праця стала підґрунтям для подальшого дослідження в напрямі проблеми парковки та пакування \cite{MathWorldRenyi}.

Задача формулювалася наступним чином: нехай є заданий відрізок $[0;~x]$, де $x>1$, і нехай на цей відрізок "паркуються" одновимірні "автомобілі" одиничної довжини, керуючись рівномірним розподілом. У такій ситуації доводилось, що середнє максимальне значення кількості машин на парковці буде $m(x)$ задовільняє наступну систему:
\begin{equation}
m(X)=
\begin{cases}
0 &$0 \leq X < 1$ \\
1 + \frac{2}{X-1} \int\limits_0^{X-1} m(y) dy &$X \geq 1$.
\end{cases}
\end{equation}

В такому випадку середня щільність автомобілів для великих $X$ виходить
\begin{equation}
m = \lim\limits_{x \rightarrow \infty} \frac{m(x)}{x} = \int\limits_0^\infty \exp\left(-2 \int\limits_0^x \frac{1-e^{-y}}{y} dy \right) dx \approx 0.747597.
\end{equation}

Але цей результат потребує узагальнення, адже описана вище класична модель Реньї не дає можливості активно застосовувати результат на практиці. Тому у ції роботі буде створена більш загальна модель, в якій водії мають змішану модель поведінки: з деякою ймовірністю вони ставлять свій автомобіль впритул до сусіднього, а в інакшому випадку – керуючись рівномірним розподілом.

У роботах Дворецького \cite{Dvoretzky} та Блезделя \cite{Blaisdell} проведено аналогічне дослідження, але отримано більш точну асимптотичну оцінку:
\begin{equation}
\label{eq:uniform_model_asymptotics_fine}
m(X) = C x - (1 - C) + \mathcal{O}\left(\left(\frac{2e}{x}\right)^{x-1.5}\right).
\end{equation}

Після отримання асимптотики для першого моменту кількості автомобілів на парковці \eqref{eq:uniform_model_asymptotics_fine} було показано асимптотичну нормальність величини
\begin{equation}
Z_{x} = \frac{N_{x} - m(x)}{\sigma(x)} \sim \operatorname{N}(0,1), \quad x \rightarrow \infty,
\end{equation}

де $N_{x}$ -- випадкова величина, що означає кількість автомобілів на парковці в термінальному стані, а $\sigma(x)$ -- стандартне відхилення $N_{x}$.

У роботі \cite{coffman1998parking} виконано аналіз моделі з пуассонівським розподілом у місці та часі.

У роботі \cite{exhaustion2017mackey} розглянуто модель паркування з трохи іншої строни --- розглядається порядок "виснаження" парковки у випадку заповнення з бінарним діленням довжини автомобіля, тобто швидкість росту зайнятого простору після заповнення проміжку автомобілями довжини 2, 1, $\frac{1}{2}, \dots$.



