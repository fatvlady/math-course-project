\section{Обґрунтування вибору напрямку дослідження}

В цитованих роботах одержані досить загальні теоретичні результати в різних моделях, які істотно узагальнюють класичну модель Реньї. В той же час, розгляд настільки загальних задач часто унеможливлює отримання конкретних констант, що з’являються в асимптотичних розкладах для різних характеристик досліджуваних моделей. Тому актуальним залишається вивчення таких узагальнень моделі Реньї, які дозволяють одержувати точні результати.

Прикладом такого узагальнення є наступна модель. Будемо вважати, що закон розподілу, згідно з яким кожний автомобіль обирає місце паркування, є не рівномірним, а сумішшю рівномірного та виродженого розподілу в лівому кінці вільного проміжку.

Наведена модель паркування допускає наступну інтерпретацію. Припустимо, що кожний водій відноситься до однієї з двох категорій – "досвідчених", частка яких становить $p$, або "недосвідчених", частка яких становить $1 - p$. Досвідчені водії намагаються припаркувати свої машини впритул до вже припаркованих раніше, в той час як недосвідчені паркуються в будь-якій точці вільного проміжку згідно з рівномірним розподілом.

Науковий інтерес представляє дослідження асимптотичної поведінки такої моделі, а також визначення старших моментів розподілу частки зайнятого проміжку.
