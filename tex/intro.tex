\likechapter{Вступ}

\todo{Review}

Історія першого автомобіля почалася ще в 1768 році разом зі створенням паросилових машин, спроможних перевозити людину. Парові машини працювали на тепловому двигуні зовнішнього згоряння, що перетворював енергію водяної пари в механічну енергію зворотно-поступального руху поршню, а потім в обертальний рух валу. Перша примітивна парова машина була побудована ще у XVII сторіччі Папеном, і являла собою циліндр з поршнем, який підіймався під дією пари, а опускався під тиском атмосфери після згущення відпрацьованої пари. Остаточні удосконалення в паровій машині були зроблені Джеймсом Уаттом в 1769 році.

В 1806 році з’явилися перші машини, які приводилися в рух двигунами внутрішнього згоряння на горючому газу, що призвело до появи в 1885 році повсюди використовуваного газолінового або бензинового двигуна внутрішнього згоряння.

Машини, що працюють на електриці ненадовго з’явилися на початку XX століття, але майже повністю зникли з поля зору аж до початку XXI століття, коли знову виникла зацікавленість в малотоксичному і екологічно чистому транспорті.

На сьогоднішній день автомобілі можна побачити усюди: надворі біля будинків, на автомагістралі чи у невеличкому провулку, в центрі міста та в селі. Автомобілів вже така незліченна кількість, що в Києві скадно знайти куточок, де не видно автотранспорту, де не чутно гулу двигунів.
Спираючись на статистику 2011 року, в Україні на 1000 осіб приходилось 158 автомобілів. А от в країнах лідерах за кількістю автомобілів, таких як, наприклад, Монако, нараховувалося порядку 900 авто на 1000 осіб.

Отже, досить гостро постає проблема організації розміщення автомобілів. Для цього створюються парковки. Але при проектуванні парковки постає проблема вибору оптимального розміру. Адже різні водії порізному паркують свої автомобілі: деякі витрачають місце економно, а деякі ставлять авто так, як їм заманеться. Тому, якщо спроектувати парковку, що може вмістити необхідну кількість автомобілів і не більше, то виникне ситуація, що місця на парковці більше немає. Якщо ж парковка буде занадто великою, то ймовірно, що багато місць будуть незайнятими, тобто простір був використаний не оптимально.

Саме тому у цій роботі розглядається задача оцінки максимальної кількості автомобілів на парковці. Для вирішення проблеми було виконано наступне:

\begin{enumerate}
	\item визначено максимальну кількість автомобілів на одновимірній парковці у крайніх, 			детермінованих випадках, а саме:
	\begin{enumerate}
		\item коли водії ставлять авто впритул до попереднього;
		\item коли водії ставлять авто рівно так, щоб між автомобілями був пропуск розміру 			майже з автомобіль, але останній туди все-таки не поміщався;
		\item коли водії ставлять свій автомобіль строго посередині вільного місця.
	\end{enumerate}
        	\item побудовано деякі моделі заповнення парковки з випадковим фактором. Визначено 		асимптотику функції залежності математичного сподівання максимальної кількості 		автомобілів на одновимірній парковці від довжини парковки для кожної побудованої моделі;
	\item створено невеликий додаток на скриптовій мові Python, що визначає коефіцієнт для 		асимптотичної оцінки математичного сподівання максимальної кількості автомобілів на 		одновимірній парковці у одновимірному випадку, а також допомагає встановити 			залежність від лінійних розмірів парковки у двовимірному випадку.
\end{enumerate}

Об'єктом дослідження є процес паркування автомобілів.

Предметом дослідження є асимптотичні властивості певних узагальнень моделі парування Реньї.

В якості методів дослідження використовуються теорія стохастичних процесів та теорія інтегральних рівнянь зі зсувом.

Наукова новизна роботи: проведено детальний аналіз результатів щодо проблеми паркування та пакування, та побудовано узагальнення класичної моделі Реньї, в якій поведінка водіїв задається не рівномірним законом розподілу, а сумішшю рівномірного та виродженого розподілів.

Практичними результатами роботи є створення додатку для моделювання процесу паркування для більш складних моделей, в тому числі і для двовимірної парковки.

Робота складається з 3 розділів. В першому розділі досліджуються існуючі підходи до вирішення задачі паркування автомобілів, надається перелік існуючої літератури. В другому розділі наводиться виведення точної асимптотичної оцінки для узагальненої моделі Реньї. У третьому розділі наведено обґрунтування вибору платформи розробки, аналіз алгоритму моделювання та алгоритму вирахування констант, отриманих у другому розділі.