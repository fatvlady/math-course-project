\likechapter{Вступ}

Проблеми послідовного пакування інтервалів, що узагальнюють відому концепцію Альфреда Реньї паркування автомобілів, мають досить широкий спектр застосувань. Це фізичні моделі структури рідин, хімічні моделі адсорбції та абсорбції, конструкція кодів з автоматичним виправленням помилок, моделювання систем комунікацій та багато інших.

Перші дослідження стосувалися задачі в класичному формулюванні Реньї. А саме: розглядалися різноманітні властивості моделі паркування автомобілів одиничної довжини та рівномірним розподілом вибору місця для автомобіля. Вивчалися асимптотичні властивості, кількості вільних проміжків довжини не меншне заданої, тощо.

Більш пізні дослідження включали вивчення узагальнених одновимірних моделей пакування зі змінними розмірами інтервалів, з он-лайн пакуванням за Пуассонівським розподілом, різними сімействами абсолютно неперервних розподілів з певними обмеженнями.

У той же час, інші дослідження стосувалися спрощеної моделі паркування автомобілів розміру 2 на вузли цілочисельної сітки у один або кілька рядів. Більш складні моделі розміщували автомобілі у вузлах випадкових дерев, або інших структурах графа.

Найбільш сучасні дослідження вивчають асимптотичну поведінку рівня заповненості у багатовимірному просторі, або ж швидкість заповнення інтервалами з експоненційно зменшуваними довжинами.

У цій роботі пропонується вивчити процес пакування одиничних інтервалів з трохі іншої сторони. Розподіл для вибору місця для наступного інтервалу задається сумішшю рівномірного та детермнованого розподілів. Для цього було:

\begin{enumerate}
	\item визначено максимальну кількість автомобілів на одновимірній парковці у крайніх, детермінованих випадках, а саме:
	\begin{enumerate}
		\item коли водії ставлять авто впритул до попереднього;
		\item вироджений випадок коли водії ставлять авто рівно так, щоб між автомобілями був пропуск розміру $1 - \varepsilon$, $\varepsilon \rightarrow 0$;
		\item коли водії ставлять свій автомобіль строго посередині вільного місця.
	\end{enumerate}
    \item побудовано модель із сумішшю рівномірного і детермінованого розподілів, для неї визначено асимптотику при достатньо великих розмірах парковки;
    \item доведено обмеженість дисперсії кількості автомобілів в момент сатурації субквадратичною функцією;
    \item виведено аналог закону великих чисел для розглянутої моделі;
	\item створено невеликий додаток на скриптовій мові Python, що визначає коефіцієнт для 	асимптотичної оцінки математичного сподівання максимальної кількості автомобілів на 	одновимірній парковці у одновимірному випадку;
    \item створено додаток на високопродуктивній мові C++, що допомагає встановити залежність від лінійних розмірів парковки у двовимірному випадку.
\end{enumerate}

Об'єктом дослідження є процес паркування автомобілів.

Предметом дослідження є асимптотичні властивості певних узагальнень моделі парування Реньї.

Практичними результатами роботи є створення додатку для моделювання процесу паркування для більш складних моделей, в тому числі і для двовимірної парковки.

Робота складається з 3 розділів. В першому розділі досліджуються існуючі підходи до вирішення задачі паркування автомобілів, надається перелік існуючої літератури. В другому розділі наводиться виведення точної асимптотичної оцінки для узагальненої моделі Реньї. У третьому розділі наведено обґрунтування вибору платформи розробки, аналіз алгоритму моделювання та алгоритму вирахування констант, отриманих у другому розділі.