\chapter*{РЕФЕРАТ}

Магістерська дисертація: \pageref*{MyLastPage}~ст., \totfig~рис., \tottab~ табл., \total{citenum}~ джерел та 3 додатки.

Об'єктом дослідження є процеси пакування і паркування Реньї.

Предметом дослідження є асимптотичні властивості певних узагальнень моделі парування Реньї.

Метою даної роботи є вивчення асимптотичних властивостей узагальненої моделі паркування і пакування Реньї з законом вибору місця вставки автомобіля, визначеного сумішшю рівномірного та виродженого розподілів.

В роботі використані методи операційного числення, методи розв'язку диференційних рівнянь зі зсувом, тауберівські теореми. Для чисельної апроксимації інтегралів були використані методи інтегрування квадратурами. Для моделювання процесів паркування було застосовано методи Монте-Карло.

Результати роботи:
\begin{itemize}
	\item побудовано інтегральне рівняння для узагальненої моделі паркування Реньї;
	\item доведено можливість застосування тауберівської теореми для визначення асимптотики математичного сподівання рівня заповненості при достатньо великих розмірах парковки;
	\item доведено асимптотичну еквівалентність математичного сподівання рівня заповненості парковки лінійній функції;
	\item виведено формулу для визначення коефіцієнта нахилу прямої, що апроксимує математичне сподівання;
	\item проведено уточнення асимптотики;
	\item доведено обмеженість другого моменту рівня заповненості лінійною функцією;
	\item виведено закон великих чисел для рівня заповненості парковки.
\end{itemize}

%Ключові слова:
\MakeUppercase{випадкові процеси, теорія ймовірностей, теорія інтегральних рівнянь зі зсувом, операційне числення, тауберові теореми.} 