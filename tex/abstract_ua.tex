\likechapternotoc{РЕФЕРАТ}

\todo{Fix this}

Дипломна робота: \pageref*{MyLastPage}~ст., \totfig~рис., \tottab~ табл., \total{citenum}~ джерел та 3 додатки.

Метою даної роботи є створення адекватної математичної моделі парковки, та отримання математичного результату у вигляді оцінки максимальної кількості автомобілів на парковці.

Результати роботи:
\begin{itemize}
	\item розглянуто 3 детерміновані моделі поведінки водіїв на парковці, виведено відповідні аналітичні формули;
	\item розглянуто 2 випадкові моделі поведінки водіїв на парковці, а саме модель рівномірного розподілу по вільному проміжку та модель суміші рівномірного розподілу та розподілу Бернуллі;
	\item реалізовано додаток консольного типу для моделювання парковки за заданою стратегією поведінки водіїв.
\end{itemize}


Новизна роботи:
\begin{itemize}
	\item проведено детальний аналіз результатів Реньї, досліджено і виведено аналітичну формулу для узагальнення класичної моделі парковки Реньї;
	\item створено додаток для імітаційного моделювання процесу парковки за заданою дисципліною паркування автомобілів.
\end{itemize}

%Ключові слова:
\MakeUppercase{випадкові процеси, теорія ймовірності, теорія інтегральних рівнянь зі зсувом, операційне числення, тауберові теореми.} 