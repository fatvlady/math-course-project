\likechapter{Висновки по роботі та перспективи подальших досліджень}

У 1958 році угорський математик Реньї порушив проблему паркування автомобілів в теоретичному аспекті. І хоча вона називається проблемою паркування, отримані результати можуть бути прикладені не лише до автомобілів. Вони можуть застосовуватися до фізичних моделей структури рідин, хімічних моделей адсорбції та абсорбції, до конструкці кодів з автоматичним виправленням помилок, до моделювання систем комунікацій та ін.

У даній магістерській дисертації було узагальнено класичну модель паркування Реньї. Отримано наступні результати.

Проведено дослідження існуючих підходів і результатів щодо проблеми паркування та пакування. Виконано аналіз аналітичного апарату, що дозволяє розв'язувати інтегральні рівняння зі зсувом, таким чином, допомагає у вирішенні проблем розглянутого класу. Було створено узагальнену модель паркування, в якій водії паркуються, керуючись сумішшю рівномірного та детермінованого розподілу. Для цього узагальнення було виведено аналітичну формулу математичного сподівання максимальної кількості автомобілів за великих розмірів парковки. Це узагальнення дає змогу використовувати модель Реньї на практиці, адже насправді існують сумлінні водії, які намагаються зайняти якомога менше місця на парковці, а є такі, що ставлять свій автомобіль як заманеться.

Паралельно з виведенням аналітичного результату був створений консольний додаток, що дозволяє моделювати процес паркування імітаційним шляхом і визначати математичне сподівання максимальної кількості автомобілів як середнє значення результатів кількох імітацій.

Було перевірено, що обидва методи дають один і той самий результат з точністю до четвертого знаку після коми на класичній моделі Реньї та на узагальненій з 10 різними ймовірностями паркування автомобіля з краю.

За аналізом отриманих результатів роботи програмного продукту можна зробити висновок, що обрані методи дозволили досягти виконання поставлених цілей з достатньо великим ступенем точності. Але у кожного метода є свої переваги та недоліки. Перевагою аналітичного методу є швидкість і досить висока точність, в той час як алгоритмічний метод дозволяє більш гнучко налаштувати поведінку водіїв на парковці.

Подальшими дослідженнями за даним напрямом можуть стати:
\begin{itemize}
\item оцінка асимптотичної поведінки моментів вищих порядків;
\item доведення центральної граничної теореми для величини $\frac{N_{p}(x) - m_{p}(x)}{\sigma_{p}(x)}$;
\item отримання аналітичного результату для асимптотики у випадку двовимірної парковки;
\item підтримка додатком змінних розмірів автомобілів;
\item створення додатку для багатовимірного розміщення.
\end{itemize}

