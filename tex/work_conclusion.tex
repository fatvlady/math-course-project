\likechapter{Висновки по роботі та перспективи подальших досліджень}

Одним з досить серйозних пороків людського виду є постійне намагання підлаштувати природу під себе. Дійсно, вже рідко де, окрім заповідників, знайдеш чисту і недоторкану природу, адже куди не глянь – усюди забудови, дороги, машини і трагічні наслідки техногенних катастроф.

В той самий час доситьактивно розвивається медицина, за останні два сторіччя істотно виріс середній вік людини, тож наша популяція стає все більше і більше. А чим більше людей, тим більше потребується житлових місць, харчів, автомобілів. І досить часто все це створюється бездумно, без розрахунку.

Саме тому у 1958 році угорський математик Реньї порушив проблему паркування автомобілів в теоретичному аспекті. І хоча вона називається проблемою паркування, результати, отримані під час її розв'язування можуть бути використані не лише до автомобілів. Їх можна застосувати для оптимального пакування, оптимального розміщення будівель, навіть для оптимальної організації робочого простору.

У даній атестаційній роботі було узагальнено класичну модель паркування Реньї. Було отримано наступні результати.

Проведено дослідження існуючих підходів і результатів щодо проблеми паркування та пакування. Проведено аналіз аналітичного апарату, що дозоляє вирішувати інтегральні рівняння зі зсувом, таким чином, допомагає у вирішенні проблем розглянутого класу. Було створено узагальнену модель паркування, в якій водії паркуються керуючись сумішшю рівномірного розподілу та розподілу Бернулі. Для цього узагальнення було виведено аналітичну формулу математичного сподівання максимальної кількості автомобілів за великих розмірів парковки. Це узагальнення дає змогу використовувати модель Реньї на практиці, адже насправді існують сумлінні водії, які намагаються зайняти якомога менше місця на парковці, а є такі, що ставлять свій автомобіль як заманеться.

Паралельно з виведенням аналітичного результату був створений консольний додаток, що дозволяє моделювати процес паркування імітаційним шляхом, і визначати математичне сподівання максимальної кількості автомобілів як середнє значення результатів кількох імітацій.

Було перевірено, що обидва методи дають один і той самий результат з точністю до четвертого знаку після коми на класичній моделі Реньї та на узагальненій з 10 різними ймовірностями паркування автомобіля зкраю.

За аналізом отриманих результатів роботи програмного продукту можна зробити висновок, що обрані методи дозволили досягти виконання поставлених цілей з достатньо великою степінню точності. Але у кожного метода є свої плюси та мінуси. Перевагою аналітичного методу є швидкість і досить висока точність, в той час як алгоритмічний метод дозволяє більш гнучко налаштувати поведінку водіїв на парковці.

Подільшими дослідженнями за даним напрямом можуть стати:
\begin{itemize}
\item отримання аналітичного результату для асимптотики у випадку двовимірної парковки;
\item зменшення складності алгоритму для двовимірної парковки;
\item підтримка додатком змінних розмірів автомобілів;
\item створення додатку для тривимірного розміщення.
\end{itemize}

