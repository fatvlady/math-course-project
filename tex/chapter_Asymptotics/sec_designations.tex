\section{Ввідні позначення}

Для початку вважатимемо, що у нас одновимірна парковка довжини $X$, на якій розташовуються автомобілі, довжини $1$ кожен.
Для спрощення будемо вважати, що водії прибувають на парковку по черзі, і залишають там свій автомобіль. Процес продовжується до того моменту, допоки парковка не заповниться. Тобто, не залишиться вільного відрізку довжини не менше за $1$.
Через $F(X)$ позначимо максимальну кількість автомобілів. Якщо $F(X)$ – випадкова величина, то через $m(X)$ будемо позначати $\EE F(X)$.
