% !TeX root=../main.tex
\section*{Висновки до розділу}
\addcontentsline{toc}{section}{Висновки до розділу}
У даному розділі було проведено асимптотичний аналіз поведінки математичного сподівання максимальної кількості автомобілів на парковці при достатньо великих лінійних розмірах парковки.

Було розглянуто узагальнену модель паркування Реньї, що задано розподілом, що є сумішшю рівномірного з вагою $q=1-p$ та виродженого розподілу з вагою $p$. Для неї за допомогою тауберівської теореми було доведено асимптотичну лінійність першого моменту $m_{p}(x)$ випадкової величини $N_{p}(x)$, що позначає кількість автомобілів на парковці довжини $x$ в момент сатурації при заповнені за вищевказаною схемою. Виконано покращення асимптотичної оцінки $m(x)$ з використанням формули Мелліна, в результаті отримано, що
\begin{equation*}
m(x) - C_{p} x - \frac{1 - C_{p}}{1 - p} \rightarrow 0, x \rightarrow \infty,
\end{equation*}

де
\begin{equation*}
C_{p} = \frac{1}{1-p} \int\limits_0^\infty \exp\left( -2\int\limits_0^u \frac{e^{\tau} - 1}{\tau(e^\tau - p)} d\tau  \right) du.
\end{equation*}

Було доведено обмеженість другого моменту $N_{p}(x)$ лінійною функцією, і доведено виконання закону великих чисел використовуючи нерівність Чебишова.

Аналітично отримані константи знаходяться чисельно, що буде виконано у наступному розділі.
