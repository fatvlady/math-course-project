% !TeX root=../main.tex
\section*{Висновки до розділу}
\addcontentsline{toc}{section}{Висновки до розділу}
У даному розділі було проведено асимптотичний аналіз поведінки математичного сподівання максимальної кількості автомобілів на парковці при достатньо великих лінійних розмірах парковки.

Було розглянуто 3 тривіальних моделі паркування автомобілів, 2 з яких є крайовими, тобто визначають верхню та нижню межу кількості автомобілів на парковці.

Було розглянуто 2 нетривіальні моделі паркування автомобілів. Для них виведено аналітичні формули асимптотичної поведінки математичного сподівання максимальної кількості автомобілів на парковці при достатньо великих розмірах парковки:
\begin{enumerate}
\item Узагальнення першої моделі. Водії з ймовірністю $p$ ставлять свій автомобіль скраю вільного проміжку, а з ймовірністю $1 - p$ – рівномірно. Отримано результат \eqref{eq:model_final}.
\end{enumerate}

Аналітично отримані константи знаходяться чисельно, що буде виконано у наступному розділі.
