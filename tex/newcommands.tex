\newcommand{\term}[1]{\textit{#1}}
\newcommand{\bydef}{\ensuremath{\stackrel{\text{\upshape df}}{=}}}
\newcommand{\<}{\langle}
\renewcommand{\>}{\rangle}
\newcommand{\includechapter}[1]{\subimport{chapter_#1/}{chapter_#1}}



\newcommand{\jointitles}{\vspace{-2.0\baselineskip}}

\newcommand{\imref}[1]{Рис.~\ref{#1}}
\newcommand{\fmref}[1]{(\ref{#1})}
\ifoptionfinal{\newcommand{\todo}[1]{}}{\newcommand{\todo}[1]{\textcolor{red}{#1}}}
\newcommand{\EE}{\mathbb{E}}
\newcommand{\DD}{\mathbb{D}}
\newcommand{\RR}{\mathbb{R}}
\newcommand{\CC}{\mathbb{C}}
\newcommand{\NN}{\mathbb{N}}
\renewcommand{\Prob}[1]{\mathsf{P}\left\{#1\right\}}
\newcommand{\Lapl}[1]{\mathcal{L}\left\{#1\right\}}

\newtheoremstyle{note}% <name>
{3pt}% <Space above>
{3pt}% <Space below>
{}% <Body font>
{}% <Indent amount>
{\upshape}% <Theorem head font>
{:}% <Punctuation after theorem head>
{.5em}% <Space after theorem headi>
{}% <Theorem head spec (can be left empty, meaning `normal')>
%\renewcommand{\proofname}{\textrm{Доведення.~}}
\renewenvironment{proof}{{\noindent\proofname.~}}{\hfill \qedsymbol \linebreak}
\theoremstyle{note}
\newtheorem{thm}{Теорема}[section]
\newtheorem{corollary}{Наслідок}[thm]
\newtheorem{lem}{Лема}[section]
\newtheorem{stmt}{Твердження}[section]
\newtheorem{defin}{Означення}[section]