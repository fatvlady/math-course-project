% !TeX root=../main.tex
\section{Аналіз роботи алгоритму}
\jointitles
\subsection{Алгоритм чисельного вирахування інтегралів}

У пункті \ref{sec:model} розділу \ref{chapter:asymptotics} було виведено формулу \eqref{eq:model_final}.
Для визначення середнього значення максимальної кількості автомобілів на парковці, керуючись отриманою формулою, необхідно чисельно визначити наступну константу:
\begin{equation}
\kappa =  \int\limits_0^\infty \exp\left( -2\int\limits_0^s \frac{1 - e^{-\tau}}{\tau} d\tau  \right) ds
\end{equation}

Оскільки чисельно рахувати інтеграл від експоненти від інтегралу незручно, неоптимально та це дасть досить велику похибку, треба звести інтеграл під експонентою до відомих функцій.
\begin{lem}
\begin{equation}
\label{eq:inner_integral}
\int\limits_0^s \frac{1 - e^{-\tau}}{\tau} d\tau = \ln s + \Gamma(0, s) + \gamma,
\end{equation}

де $\gamma=0.5772156649$ – константа Ейлера-Маскероні, $\Gamma(0,s)$ – неповна гамма-функція.
\end{lem}
\begin{proof}
\begin{gather*}
	\int\limits_0^s \frac{1 - e^{-\tau}}{\tau} d\tau = \int\limits_0^1 \frac{1 - e^{-\tau}}{\tau} d\tau  + \int\limits_1^s \frac{1}{\tau} d\tau + \int\limits_s^\infty \frac{e^{-\tau}}{\tau} d\tau - \int\limits_1^\infty \frac{e^{-\tau}}{\tau} d\tau = \\
	= \int\limits_0^1 \frac{1 - e^{-\tau}}{\tau} d\tau  + \ln s + \Gamma(0,s) -\int\limits_1^\infty \frac{e^{-\tau}}{\tau} d\tau = \\
	= \ln s + \Gamma(0,s) + \lim\limits_{a \rightarrow 0} \left. (1-e^{-\tau}) \ln \tau \right|_0^1  - \int\limits_0^1 e^{-\tau} \ln \tau d\tau - \\
	- \underbrace{\left. e^{-\tau} \ln \tau \right|_1^\infty}_0 - \int\limits_1^\infty e^{-\tau} \ln \tau d\tau = \ln s + \Gamma(0,s) +  \\
	+ \lim\limits_{a \rightarrow 0} \left. (1-e^{-\tau}) \ln \tau \right|_0^1 - \int\limits_0^\infty e^{-\tau} \ln \tau d\tau = \lim\limits_{a \rightarrow 0} \left. (1-e^{-\tau}) \ln \tau \right|_0^1 + \\
	+ \ln s + \Gamma(0,s) + \gamma.
\end{gather*}

Останній перехід випливає з тотожності
$$
\gamma = - \int\limits_0^\infty e^{-\tau} \ln \tau d\tau.
$$

Якщо довести, що $\lim\limits_{a \rightarrow 0} \left. (1-e^{-\tau}) \ln \tau \right|_0^1 = 0$, то лему доведено.

З озкладу у ряд Тейлора випливає, що
$$
	\lim\limits_{\tau \rightarrow 0}  \frac{1-e^{-\tau}}{ \tau} = 1.
$$

Тому 
\begin{gather*}
	\lim\limits_{\tau \rightarrow 0} (1-e^{-\tau}) \ln \tau = \lim\limits_{\tau \rightarrow 0} \frac{1-e^{-\tau}}{ \tau} \cdot \frac{\ln u}{u^{-1}} = \\
	=\<\text{правило Лопіталя}\> = 1 \cdot \lim\limits_{\tau \rightarrow 0} \frac{u^{-1}}{-u^{-2}} = 0.
\end{gather*}

Тому $\lim\limits_{a \rightarrow 0} \left. (1-e^{-\tau}) \ln \tau \right|_0^1 = 0$, а отже, лему доведено.
\end{proof}

Таким чином отримали, що
\begin{equation}
\kappa = e^{-2\gamma} \int\limits_0^\infty \frac{\exp(-2\Gamma(0,s))}{s^2} ds.
\end{equation}

Для підрахунку невласного інтегралу можна оцінити його залишок, і знайти такі межі інтегрування, щоб залишок не перевищував деякого $\varepsilon$.

Тобто необхідно знайти таке $\delta$, щоб виконувалось
$$
	e^{-2\gamma} \int\limits_\delta^\infty \frac{\exp(-2\Gamma(0,s))}{s^2} ds < \varepsilon.
$$

Оскільки $\exp(-2\Gamma(0,s)) < 1$, то достатньо знайти таке $\delta$, щоб виконувалось
$$
	\int\limits_\delta^\infty \frac{1}{s^2} < \varepsilon e^{2\gamma}.
$$

Оскілки беремо $\delta \geq 0$, то
\begin{align*}
	\int\limits_\delta^\infty \frac{1}{s^2} &< \varepsilon e^{2\gamma},\\
	-\left. \frac{1}{s} \right|_\delta^\infty &< \varepsilon e^{2\gamma}, \\
	\delta^{-1} &< \varepsilon e^{2\gamma},  \\
	\varepsilon^{-1} e^{-2\gamma} &< \delta.
\end{align*}

Тобто, щоб отримати результат з точністю $10^{-6}$, треба взяти  $\delta > 10^6 e^{-2\gamma} \approx 315236$. Це і реалізовано в додатку "integral".

\subsection{Алгоритм моделювання на одновимірній парковці}

Для моделювання одновимірної парковки виконується наступний алгоритм дій:
\begin{enumerate}[label={\arabic*.}]
\item Створюється вектор довжин вільних проміжків, який ініціалізується однією довжиною – довжиною парковки. 
\item Допоки вектор не пустий, виконується: 
\begin{enumerate}[label={2.\arabic*.}]
\item Вибирається поведінка водія використовуючи рандомізатор.
\item Дістається остання довжина з вектору, і ділиться на частини відповідно до моделі поведінки водія.
\item Кожна з частин додається до вектору, якщо її довжина не менша 1, тобто, довжини автомобіля.
\item Інкрементується поточна кількість автомобілів.
\end{enumerate}
\end{enumerate}

Цей процес повторюється задану кількість разів, і в кінці програма видає середнє значення автомобілів з усіх ітерацій, а також відношення до довжини парковки.

\subsection{Алгоритм моделювання на двовимірній парковці}

Двовимірний алгоритм працює аналогічно одновимірному, тільки розглядаються не проміжки, а прямокутники. А саме, створюється список вільних прямокутників. Під вільним розуміється прямокутник, в якому будь-яка точка може бути потенційним центром автомобіля.

Ініціалізується цей список початковим прямокутником – $(\frac{1}{2}, \frac{1}{2}, a - \frac{1}{2}, b - \frac{1}{2})$, де $a,~b$ – задані лінійні розміри парковки. Спираючись на площі поточних прямокутників, обирається один з ймовірністю рівною відношенню його площі до сумарної площі усіх вільних прямокутників. Після цього всередині обраного прямокутника обирається за рівномірним розподілом точка, навколо якої будується квадратний окіл розміру 2x2, і перевіряється на перетин з усіма поточними вільними прямокутниками. Якщо перетинається, то вільний прямокутник ділиться на менші прямокутники, при чому один із прямокутників – область перетину. Усі менші прямокутники, окрім перетину, додаються в список.
