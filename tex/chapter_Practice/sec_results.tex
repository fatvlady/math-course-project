\section{Результати роботи програми}
\label{sec:result_analyse}
\jointitles
\subsection{Результати підрахунку констант}
Для формули \eqref{eq:model_final} було чисельно пораховано коефіцієнт $\kappa_\alpha$, для кожного $\alpha$ від 0 до 0.9 з кроком 0.1. Результати наведені в таблиці~\ref{tab:model_integral_result}.

\begin{table}[ht]
	\caption{Результати розрахунку константи для узагальненої моделі}
	\centering
\begin{tabular}{|p{0.14\textwidth}|p{0.14\textwidth}|p{0.14\textwidth}|p{0.14\textwidth}|p{0.14\textwidth}|p{0.14\textwidth}|}
	\hline
	$\alpha$ & 0 & 0.1 & 0.2 & 0.3 & 0.4 \\
	\hline
	$\kappa_\alpha$ & 0.747598 &0.76351 &0.780574 &0.798962 &0.818896  \\
	\hline
	\hline
	$\alpha$  & 0.5 & 0.6 & 0.7 & 0.8 & 0.9\\
	\hline
	$\kappa_\alpha$  &0.84066 &0.864638 &0.891365 &0.92165 &0.956849 \\
	\hline
\end{tabular}	
	\label{tab:model_integral_result}
\end{table}

\subsection{Результати роботи моделера}

Для формули \eqref{eq:model_final} було промодельовано поведінку водіїв і емпірично визначено коефіцієнт $\kappa_\alpha$, для кожного $\alpha$ від 0 до 0.9 з кроком 0.1. Результати наведені в таблиці~\ref{tab:model_simulation_result}.

\begin{table}[ht]
	\caption{Результати розрахунку константи для узагальненої моделі}
	\centering
\begin{tabular}{|p{0.14\textwidth}|p{0.14\textwidth}|p{0.14\textwidth}|p{0.14\textwidth}|p{0.14\textwidth}|p{0.14\textwidth}|}
	\hline
	$\alpha$ & 0 & 0.1 & 0.2 & 0.3 & 0.4 \\
	\hline
	$\kappa_\alpha$ & 0.747588 &0.76352 &0.780569 &0.798959 &0.818891  \\
	\hline
	\hline
	$\alpha$  & 0.5 & 0.6 & 0.7 & 0.8 & 0.9\\
	\hline
	$\kappa_\alpha$  &0.84055 &0.863658 &0.891214 &0.92158 &0.95693 \\
	\hline
\end{tabular}	
	\label{tab:model_simulation_result}
\end{table}

Нескладно помітити, що теоретичний результат співпав з результатом моделювання з точністю до 4 знаку.

Для двовимірного випадку існує припущення, що відношення середньої кількості автомобілів до загальної площі парковки прямує до $\kappa^2 \approx 0.56$, але це не доведений факт \cite{MathWorldRenyi}. Було проведено експеримент на розмірах 50x50, 100x100, 200x200, і результати вийшли відповідно 0.7, 0.6 та 0.58, тобто ймовірно, що припущення правдиве, але для перевірки на дуже великих розмірах парковки необхідні дуже потужні обчислювальні ресурси через велику складність двовимірного алгоритму.
