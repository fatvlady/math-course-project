\section[Вибір платформи і мови реалізації]{Обґрунтування вибору платформи та мови реалізації програмного продукту}

Реалізація програмного продукту проведена за допомогою мови C++ та Python. Вибір такого тандему є цілком виправданим. Мова C++ є мовою середнього рівня - в ній присутні елементи мов програмування як низького (підмножина - C), так і високого рівня, що робить С++ дуже ефективною для розробки складних проектів. При цьому виконання елементарних дій (попіксельна обробка зображень) та алгоритмів вцілому залишається максимально швидкою. 

Мова Python використовується для швидких математичних розрахунків, в тому числі, для вирахування чисельних значень інтегралів.

Основною платформою для використання програмного продукту було вибрано сімейство операційних систем на базі UNIX, оскільки процес розробки та використання додатків з консольним інтерфейсом там найбільше спрощений. Також, для швидких математичних розрахунків необхідне середовище виконання Python, що дуже просто налаштовується в системах сімейства Linux та в більшості випадків присутнє в стандартній комплектації операційної системи.

Однак, розробка програмного продукту проводилася з дотриманням правил кросплатформеного програмування, що дає можливість збирати виконувані файли і в інших популярних операційних системах (наприклад, сімейства Windows NT).