% !TeX root=main.tex
\usepackage{ifdraft}

\usepackage{fontspec}
\usepackage{xltxtra} 
\usepackage{polyglossia}   %% загружает пакет многоязыковой вёрстки

\setdefaultlanguage{ukrainian}  %% устанавливает главный язык документа
\setotherlanguage{english} %% объявляет второй язык документа
\setotherlanguage{russian} %% третий язык документа - для библиографии
\defaultfontfeatures{Ligatures=TeX,Mapping=tex-text}

%% Виставлення шрифтів документу
\newfontfamily{\cyrillicfont}{Times New Roman}
\newfontfamily{\cyrillicfontsf}{Arial}
\newfontfamily{\cyrillicfonttt}[Scale=0.95]{Courier New}

\usepackage{extsizes}
\usepackage{titlesec}
\usepackage{geometry}
	\geometry{left=3cm}
	\geometry{right=1.5cm}
	\geometry{top=2.4cm}
	\geometry{bottom=2.4cm}

\usepackage{fancyhdr}
	\pagestyle{fancy}
	\fancyhf{}
	\fancyhead[R]{\thepage}
	\fancyheadoffset{0mm}
	\fancyfootoffset{0mm}
	\setlength{\headheight}{17pt}
	\renewcommand{\headrulewidth}{0pt}
	\renewcommand{\footrulewidth}{0pt}
	\fancypagestyle{plain}{
		\fancyhf{}
		\rhead{\thepage}}

\usepackage{setspace}%[nodisplayskipstretch]
	\onehalfspacing
	\frenchspacing

%%% Вирівнювання та переноси %%%
\sloppy
\clubpenalty=10000
\widowpenalty=10000

\usepackage{textcomp}
\usepackage{gensymb}

\usepackage{graphicx} 	% для вставки картинок


\usepackage{amsfonts,amsmath,amsthm} % математические дополнения от АМС
\usepackage{unicode-math}
\usepackage{indentfirst} % отделять первую строку раздела абзацным отступом тоже
\usepackage[usenames,dvipsnames]{color} % названия цветов
\usepackage{ulem} % подчеркивания
\usepackage{tocloft}
\usepackage{import}
\usepackage{lastpage} % знаходить номер останньо\UTF{0457} сторінки за \pageref*{LastPage}~с.
\usepackage{etoolbox}
\usepackage[title,titletoc]{appendix}
\usepackage{pdfpages}

\usepackage{pgfplots}

\usepackage{listings}
\lstset{
	basicstyle=\footnotesize\ttfamily,        % the size of the fonts that are used for the code
	breakatwhitespace=true,         % sets if automatic breaks should only happen at whitespace
	breaklines=true,                 % sets automatic line breaking
	commentstyle=\color{Green},	    % comment style
	%	extendedchars=true,              % lets you use non-ASCII characters; for 8-bits encodings only, does not work with UTF-8
	keepspaces=true,                 % keeps spaces in text, useful for keeping indentation of code (possibly needs columns=flexible)
	keywordstyle=\color{Blue},       % keyword style
	language=C++, 	                % the language of the code 
	%TODO:add c++11
	numbers=left,                 	   % where to put the line-numbers; possible values are (none, left, right)
	numbersep=7pt,                  	 % how far the line-numbers are from the code
	numberstyle=\tiny\color{Gray}, 		% the style that is used for the line-numbers
	showtabs=false,                  % show tabs within strings adding particular underscores
	stepnumber=5,                    % the step between two line-numbers. 
	stringstyle=\color{OliveGreen},     % string literal style
	tabsize=2,                       % sets default tabsize to 2 spaces
	title=\lstname                   % show the filename of files included with \lstinputlisting; also try caption instead of title
}

%%% Форматування підписів таблиць та рисунків %%%
\usepackage{ragged2e}
\usepackage{caption}
\usepackage{subcaption}
	\DeclareCaptionLabelFormat{gostfigure}{Рисунок #2}
	\DeclareCaptionLabelFormat{gosttable}{Таблиця #2}
	\DeclareCaptionLabelSeparator{gost}{~--~}
	\captionsetup{labelsep=gost}
	\captionsetup[table]{position=top,labelformat=gosttable,justification=justified,singlelinecheck=false, margin={1.25cm,0cm}}
	\captionsetup[figure]{position=bottom, labelformat=gostfigure,justification=centering}
	\renewcommand{\thesubfigure}{\asbuk{subfigure}}

%пакети для таблиць
\usepackage{array}
\usepackage{tabulary}
\usepackage{tabularx}
\usepackage{float} 		% пакет для [H]-положення фігур
\usepackage{rotating}	% пакет для перевертання блоків
\usepackage{makecell}
\usepackage{multirow} % улучшенное форматирование таблиц

%пакет для налаштування сторінки

\titleformat{\chapter}[block]
{\filcenter}
{\MakeUppercase{\bfseries\chaptertitlename} \thechapter}
{8pt}
{\bfseries}{}

\titleformat{\section}
{\normalsize\bfseries}
{\thesection}
{1em}{}

\titleformat{\subsection}
{\normalsize\bfseries}
{\thesubsection}
{1em}{}

\titleformat{\subsubsection}
{\normalsize\bfseries}
{\thesubsubsection}
{1em}{}

\titleformat{\paragraph}
{\normalsize\bfseries}
{\thesubsubsection}
{1em}{}

\titleformat{\subparagraph}
{\filcenter}
{\MakeUppercase{\chaptertitlename} \thechapter.}
{0.5em}
{\bfseries}{}

\titlespacing*{\chapter}{0pt}{0pt}{*8}
\titlespacing*{\section}{\parindent}{*8}{*8}
\titlespacing*{\subsection}{\parindent}{*8}{*8}
\titlespacing*{\subsubsection}{\parindent}{*8}{*8}
\titlespacing*{\paragraph}{\parindent}{*1}{*1}
\titlespacing*{\subparagraph}{\parindent}{*4}{*2}

\setlength{\parindent}{1.25cm} 

% table of contents setup
\renewcommand{\cfttoctitlefont}{\hspace{0.38\textwidth} \bfseries\MakeUppercase}
\renewcommand{\cftbeforetoctitleskip}{-1em}
\renewcommand{\cftaftertoctitle}{\vspace{2em}\mbox{}\hfill \\ \mbox{}\hfill{\footnotesize Ст.}\vspace{-2.5em}}
\renewcommand{\cftchapfont}{\normalsize\bfseries \MakeUppercase{\chaptername} }
\renewcommand{\cftsecfont}{\hspace{31pt}}
\renewcommand{\cftsubsecfont}{\hspace{11pt}}
\renewcommand{\cftbeforechapskip}{1em}
\renewcommand{\cftparskip}{-1mm}
\renewcommand{\cftdotsep}{1}
\setcounter{tocdepth}{2}
\setcounter{secnumdepth}{5}

\usepackage[square,numbers,sort&compress]{natbib}
\renewcommand{\bibnumfmt}[1]{#1.\hfill}
\renewcommand{\bibsection}{\likechapter{Список літератури}}
\setlength{\bibsep}{0pt}

%% блок роботи з кількістю рисунків, таблиць і книжок
\providecommand\totfig{} 
\providecommand\tottab{}

\newcounter{totfigures}
\newcounter{tottables}
\newcounter{totreferences}
\makeatletter
\renewcommand{\@dotsep}{2}
\newcommand{\l@likechapter}[2]{{\bfseries\@dottedtocline{0}{0pt}{0pt}{#1}{#2}}}
\AtEndDocument{%
	\addtocounter{totfigures}{\value{figure}}%
	\addtocounter{tottables}{\value{table}}%
	\immediate\write\@mainaux{%
		\string\gdef\string\totfig{\number\value{totfigures}}%
		\string\gdef\string\tottab{\number\value{tottables}}%
		\string\gdef\string\totref{\number\value{totreferences}}%
	}%
}
\makeatother
\pretocmd{\chapter}{\addtocounter{totfigures}		{\value{figure}}}	{}{}
\pretocmd{\chapter}{\addtocounter{tottables}		{\value{table}}}	{}{}
\pretocmd{\bibitem}{\addtocounter{totreferences}	{1}}				{}{}

\makeatletter
\bibliographystyle{gost}   % Оформляем библиографию по дсту
\renewcommand{\@biblabel}[1]{#1.} % Заменяем библиографию с квадратных скобок на точку
\makeatother

\usepackage{totcount} 	% робота з підрахунком бібліографічних посилань
	\newtotcounter{citenum}
	\def\oldcite{}
	\let\oldcite=\bibcite
	\def\bibcite{\stepcounter{citenum}\oldcite}

\newcommand{\likechapterheading}[1]{
	\begin{center}
		\par\textbf{\MakeUppercase{#1}}
	\end{center}
	\noindent\ignorespaces
}
\newcommand{\likechapter}[1]{	
	\phantomsection
	\likechapterheading{#1}	
	\addcontentsline{toc}{likechapter}{\texorpdfstring{\MakeUppercase{#1}}{#1}}}
\newcommand{\likechapternotoc}[1]{
	\phantomsection
	\likechapterheading{#1}}


\newcommand{\empline}{\mbox{}\newline}

\newcommand{\append}[1]{
	\clearpage
	\stepcounter{chapter}	
	\subparagraph{\MakeUppercase{#1}}
	\addcontentsline{toc}{likechapter}{\texorpdfstring{\MakeUppercase{\chaptertitlename~\Asbuk{chapter}\;#1}}{\chaptertitlename~\Asbuk{chapter}:~#1}}}

\usepackage{enumitem}
	\makeatletter
	\AddEnumerateCounter{\asbuk}{\@asbuk}{м)}
	\makeatother
	\setlist{nolistsep}
	\renewcommand{\labelitemi}{-}
	\renewcommand{\labelenumi}{\asbuk{enumi})}
	\renewcommand{\labelenumii}{\arabic{enumii})}

\usepackage[xetex,bookmarks=true,colorlinks=true,linkcolor=MidnightBlue,citecolor=Green,linktoc=none,urlcolor=Green]{hyperref}
	\renewcommand\UrlFont{\ttfamily}%
	
\makeatletter
\g@addto@macro\normalsize{%
	\setlength\abovedisplayskip{20pt plus 3pt minus 3 pt}
	\setlength\belowdisplayskip{20pt plus 3pt minus 3 pt}
	\setlength\abovedisplayshortskip{20pt plus 3pt minus 3 pt}
	\setlength\belowdisplayshortskip{20pt plus 3pt minus 3 pt}
}
\makeatother