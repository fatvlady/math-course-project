\begin{frame}{Постановка задачі}
	\manimate
	\vspace{-8pt}
	\begin{block}{Мета роботи}
		 Дослідження асимптотики максимальної кількості автомобілів, які можуть бути розміщені на паркінгу, в залежності від величини паркінгу
	\end{block} 
	\note[item]{Таким чином, метою роботи є виведення математичного результату.}

	\begin{block}{Об'єкт дослідження}
		Процес випадкового розміщення автомобілів на паркінгу
	\end{block}
	\note[item]{Об'єктом дослідження є процес випадкового розміщення автомобілів на паркінгу}
	
	\begin{block}{Предмет дослідження}
		Методи стохастичного моделювання процесу паркування автомобілів
	\end{block}
	\note[item]{Предметом дослідження є методи стохастичного моделювання процесу паркування автомобілів.}
\end{frame}

\begin{frame}{Постановка задачі}
	\manimate
	\vspace{-8pt}
		\begin{block}{Поставлені задачі}
			\begin{itemize}
				\item дослідження асимптотики математичного сподівання максимальної кількості автомобілів, які можуть бути розміщені на паркінгу в залежності від величини паркінгу
				\begin{itemize}
					\item у випадку рівномірного розподілу місця паркування
					\item у випадку суміші рівномірного розподілу і розподілу Бернуллі
				\end{itemize}
				\item розробка та комп'ютерна реалізація алгоритму імітаційного моделювання процесу заповнення паркінгу в одновимірному та двовимірному випадках
			\end{itemize}
		\end{block}
\end{frame}
\note{У рамках цієї роботи були поставлені наступні задачі:(...)

}