\begin{frame}{Постановка задачі}
	\vspace{-8pt}
	\begin{block}{Мета роботи}
		 Вивчення асимптотичних властивостей узагальненої моделі і паркування і пакування Реньї
	\end{block} 
	\note[item]{Метою даної роботи є вивчення асимптотичних властивостей узагальненої моделі паркування і пакування Реньї з законом вибору місця вставки автомобіля, визначеного сумішшю рівномірного та виродженого розподілів.}

	\begin{block}{Об'єкт дослідження}
		Процеси пакування і паркування Реньї
	\end{block}
	\note[item]{Об’єктом дослідження є процеси пакування і паркування Реньї.}
	
	\begin{block}{Предмет дослідження}
		Асимптотичні властивості певних узагальнень моделі парування Реньї
	\end{block}
	\note[item]{Предметом дослідження є Асимптотичні властивості певних узагальнень моделі парування Реньї.}
\end{frame}

\begin{frame}{Постановка задачі}
	\begin{block}{Узагальнена модель паркування}
	\begin{enumerate}
		\item Заданий початковий відрізок довжини $x$.
		\item Задано сімейство розподілів $U_{a,b}$ з такою функцією розподілу $F_{U}(t)$, що $F_{U}(a)=0$ і $F_{U}(b)=1$.
		\item Допоки існує хоча б один незаповнений інтервал $[a_i, b_i]$ такий, що $b_i-a_i \geq 1$:
		\begin{enumerate}
			\item З розподілу $U_{a_i,b_i - 1}$ обирається випадкове значення $t$ -- позиція лівого краю одиничного інтервалу.
			\item На початковому відрізку інтервал $(t, t+1)$ позначається як заповнена множина.
		\end{enumerate}
	\end{enumerate}
	\end{block}
\end{frame}

\begin{frame}{Постановка задачі}
	У класичній моделі паркування у якості $U_{a,b}$ виступає рівномірний розподіл на інтервалі $(a,b)$. Пропонується розглянути узагальнену модель, в якій  $U_{a,b}=U_{p}(a,b)$ задається функцією розподілу
	\begin{equation*}
	F_{U_{p}(a,b)}(t)=p \cdot [t \geq a] + (1-p)\cdot\begin{cases}
	0, &t<a,\\
	\frac{t-a}{b-a}, &t \in[a,b],\\
	1, &t > b,
	\end{cases}
	\end{equation*}
	де $p \in [0,1)$ -- фіксоване.
\note{
}
\end{frame}

\begin{frame}{Постановка задачі}
	Досліджується кількість інтервалів $N(x)$, розміщених на відрізку в момент сатурації.
	Цілі дослідження:
	\begin{itemize}
		\item визначення асимптотики $m_{p}(x) = \mathbb{E}N(x)$;
		\item доведення асимптотичної субквадратичності другого центрального моменту величини $N(x)$ на нескінченності;
		\item виведення закону великих чисел для $\frac{N(x)}{x}$;
		\item розробка та комп'ютерна реалізація алгоритму імітаційного моделювання процесу заповнення паркінгу в одновимірному та двовимірному випадках.
	\end{itemize}
	\note{
	}
\end{frame}