\begin{frame}{Висновки}
        	В роботі, використовуючи аналітичні та чисельні методи, досліджено асимптотику математичного сподівання максимальної кількості автомобілів, що можуть бути припарковані на стоянці, за різних припущень щодо вибору місця паркування. Для випадкового вибору цього місця на вільній ділянці розглядалося два ймовірнісні розподіли – рівномірний та суміш рівномірного розподілу та розподілу Бернуллі. Другий розподіл, очевидно, є узагальненням першого та дозволяє врахувати наявність водіїв з різним досвідом паркування.

\end{frame}

\begin{frame}{Висновки}
	В обох випадках отримано вказану асимптотику та підтверджено її методами імітаційного моделювання. Для цього було розроблено та реалізовано чисельний алгоритм, що дозволяє з високою точністю оцінити шукані характеристики.
      
        Крім того, в роботі було проведено імітаційне моделювання випадкового паркування автомобілів на двовимірному прямокутному паркінгу.

\end{frame}

\begin{frame}{Шляхи подальшого розвитку}
	\begin{itemize}
		\item знаходження асимптотики інших числових характеристик максимальної кількості припаркованих автомобілів (дисперсія, старші моменти тощо);
		\note[item]{дозволить краще зрозуміти поведінку максимальної кількості припаркованих автомобілів}
		\item формулювання та доведення граничних теорем про слабку збіжність розподілу відповідним чином нормованої максимальної кількості припаркованих автомобілів;
		\note[item]{виведення аналогу ЦГТ для парковки}
		\item аналіз моделей з іншими припущеннями щодо розподілу випадкового вибору місця паркування. 
		\note[item]{більше узагальнення.}
	\end{itemize}
\end{frame}

\begin{frame}
	\centering
	\Large Дякую за увагу.

\end{frame}