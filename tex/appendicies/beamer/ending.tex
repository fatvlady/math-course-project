\begin{frame}{Висновки}
	В роботі, використовуючи аналітичні та чисельні методи, досліджено асимптотику першого та другого моментів випадкової величини, що позначає кількість автомобілей на парковці в момент виснаження. Було доведено, що математичне сподівання збігається до лінійної функції зі швидкістю, більшою за поліноміальну. Отримано явний вигляд коефіцієнтів вищевказаної лінійної функції, хоч і у досить складній репрезентації подвійного інтегралу.
\end{frame}

\begin{frame}{Висновки}
	Отриману асимптотичну оцінку першого моменту було застосовано для доведення обмеженості другого центрального моменту функцією, що повільніша за квадратичну на нескінченності. Це дозволило використати нерівність Чебишова для виведення закону великих чисел.
	
	Окрім математичного результату, було створено консольний додаток для імітаційного моделювання процесу паркування з виснаженням. За допомогою нього було перевірено теоретичний результат.
\end{frame}

\begin{frame}{Шляхи подальшого розвитку}
	\begin{itemize}
		\item оцінка асимптотичної поведінки моментів вищих порядків;
		\item доведення центральної граничної теореми для величини $\frac{N_{p}(x) - m_{p}(x)}{\sigma_{p}(x)}$;
		\item отримання аналітичного результату для асимптотики у випадку двовимірної парковки;
		\item підтримка додатком змінних розмірів автомобілів;
		\item створення додатку для моделювання багатовимірного розміщення.
	\end{itemize}
\end{frame}

\begin{frame}
	\centering
	\Large Дякую за увагу.

\end{frame}