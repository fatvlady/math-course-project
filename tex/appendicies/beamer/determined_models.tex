\begin{frame}{Детерміновані моделі}
	\manimate
	\vspace{-20pt}
	\centering
	\begin{figure}   
		\centering
		\includegraphics[width=1\linewidth, height=20pt]{im/parking_trivial1}
		\caption{"Правильні" водії}
		\note[item]{ Водії стають впритул один до одного.}
	\end{figure}
	\begin{figure}   
		\centering
		\includegraphics[width=1\linewidth, height=20pt]{im/parking_trivial2}
		\caption{"Неправильні" водії}
		\note[item]{ Водії намагаються загарбати якомога більше простору.}
	\end{figure}
	\begin{figure}
		\centering
		\includegraphics[width=1\linewidth, height=20pt]{im/parking_trivial3}
		\caption{Автомобілі стають по центру вільного проміжку}
		\note[item]{ Водії ставлять автомобіль посередині вільного проміжку (показати)}
	\end{figure}
\end{frame}



\note{
 Перша модель - верхня межа, друга - нижня межа. Третя модель є прикладом моделі без асимптотичної поведінки.
}