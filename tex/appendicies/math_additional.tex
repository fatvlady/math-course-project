\append{Теоретичний мінімум}

\section{Поняття випадкової величини}
Одним з найважливішим поняттям в теорії ймовірностей є поняття випадкової величини. Під випадковою величиною розуміють змінну, яка в результаті дослідження в залежності від випадку приймає одне з можливої множини своїх значень (яке саме – заздалегідь невідомо).

\begin{defin}
	Випадковою величиною $X$ називається функція, що задана у просторі елементарних подій $\Omega$, тобто $X \in f(\omega)$, де $\omega$ – елементарна подія, що належить простору $\Omega$ \cite{Kremer}.
\end{defin}

Для дискретної випадкової величини множина можливих значеньфункції $f(\omega)$ скінчена або злічена, для неперервної – нескінченна або незлічена.

\begin{defin}
	Законом розподілу випадкової величини називається відношення, що встановлює зв'язок між можливим значеннями випадкової величини і відповідними їм ймовірностями \cite{Kremer}.
\end{defin}

\subsection{Представлення випадкових величин}
\begin{defin}
	Функцією розподілу випадкової величини $X$ називається функція $F(X)$, котра виражає для кожного $x$ ймовірність того, що випадкова величина $X$ прийме значення, яке менше за $x$ \cite{Kremer}:
	\begin{equation}
		F(x)=\Prob{X < x}
	\end{equation}
\end{defin}

\begin{defin}
	Випадкова величина називається дискретною, якщо вона набуває скінчену або злічену множину значень.
\end{defin}

\begin{defin}
	Випадкова величина називається неперервною, якщо її функція розподілу є неперервною, диференційованою майже скрізь, за винятком, можливо, окремих ізольованих точок \cite{Kremer}.
\end{defin}

Функцію $F(X)$ іноді називають інтегральною функцією розподілу або інтегральним законом розподілу. Визначення неперервної випадкової величини за допомогою функції розподілу не є єдиним.

\begin{defin}
Щільністю ймовірності (щільністю розподілу або просто щільністю) $f(x)$ неперервної випадкової величини $X$ називається похідна її функції розподілу $f(x)$ = $F'(x)$ \cite{Kremer}.
\end{defin}

Щільність ймовірності іноді називають диференціальною функцією розподілу або диференціальним законом розподілу. А графік $f(x)$ – кривою розподілу.

\subsection{Числові характеристики випадкових величин}
\begin{defin}
	Математичним сподіванням, або середнім значенням $\EE(X)$ дискретної випадкової величини X називається сума добутків всіх його значень на відповідні їм ймовірності \cite{Kremer}:
	\begin{equation}
		\EE(X)= \sum\limits_{i=1}^n x_i \cdot p_i
	\end{equation}
\end{defin}

\begin{defin}
	Математичне сподівання неперервної випадкової величини визначаться \cite{Kremer}:
	\begin{equation}
		\EE(X)= \int\limits_{-\infty}^\infty x f(x) dx
	\end{equation}
\end{defin}

\begin{defin}
	Дисперсією $\DD(X)$ випадкової величини $X$ називається математичне сподівання квадрата її відхилення від математичного сподівання: $\DD(X)=\EE(X-\EE X)^2$. Або $\DD(X)=\EE(X - a)^2$, ~де $a=\EE X$.
	
	Якщо $X$ – дискретна, то
	\begin{equation}
		\DD(X)=\sum\limits_{i=1}^n (x_i - \EE X)^2 \cdot p_i
	\end{equation}
	
	Якщо $X$ – неперервна, то
	\begin{equation}
		\DD(X)=\int\limits_{-\infty}^\infty (x - \EE X)^2 f(x) dx
	\end{equation}
\end{defin}

\begin{defin}
	Середнім квадратичним відхиленням (стандартним відхиленням або стандартом) $\sigma_X$ випадкової величини X називається арифметичним значенням кореня квадратного з дисперсії: $\sigma_X = \sqrt{\DD X}$
\end{defin}

Математичне сподівання $\EE X$, або перший початковий момент, характеризує середнє значення або положення розподілу випадкової величини $X$ на числовій осі; дисперсія $\DD X$, або другий центральний момент $\mu_2$, – величину розсіювання розподілу $X$ відносно $\EE X$ \cite{gmurman1977theory}.

\section{Перетворення Лапласа}

Перетворення Лапласа - інтегральне перетворення, що зв'язує функцію $F(s)$ комплексної змінної (зображення) з функцією $f(t)$ дійсного змінного (оригінал). З його допомогою досліджуються властивості динамічних систем і вирішуються диференціальні і інтегральні рівняння.

Однією з особливостей перетворення Лапласа, які визначили його широке поширення в наукових і інженерних розрахунках, є те, що багатьом співвідношенням і операціям над оригіналами відповідають простіші співвідношення над їх зображеннями. Так, згортка двох функцій зводиться в просторі зображень до операції множення, а лінійні диференціальні рівняння стають алгебраїчними.

\begin{defin}
	Перетворенням Лапласа функції дійсної змінної $f(t)$ називається функція $F(s)$ комплексної змінної $s=\theta + i\omega$ така, що:
	\begin{equation}
		F(s)=\Lapl{f(t)}=\int\limits_0^\infty e^{-st} f(t) dt
	\end{equation}
	при цьому права частина виразу називається інтегралом Лапласа \cite{mclachlan2014laplace}.
\end{defin}

Перетворення Лапласа існує, якщо $f(t): \RR^+ \cup {0} \rightarrow \RR$ – інтегровна на будь-якому підінтервалі $[0, +\infty)$ і виконується нерівність $|f(t)| < K e^{\omega t}$ для деяких фіксованих $K > 0, ~\omega ≥ 0$.

\subsection{Властивості перетворення Лапласа}

\begin{enumerate}[label=\arabic*.]
	\item Перетворення Лапласа лінійне, тобто для $\forall \alpha, \beta \in \CC$
            	\begin{equation}
            		\Lapl{𝛼f(t) + 𝛽g(t)} = 𝛼𝐹(𝑠) + 𝛽𝐺(𝑠)
            	\end{equation}
	\item Теорема подібності. Для $\forall \alpha \in \RR^+$
		\begin{equation}
			\Lapl{f(\alpha t)} = \frac{1}{\alpha} F(\alpha s)
		\end{equation}
	\item Диференціювання оригінала. Якщо перетворення Лапласа існує для $f′(t)$, то
		\begin{equation}
			\Lapl{f'(t)} = sF(s) - f(0)
		\end{equation}
	\item Диференціювання зображення. Зводиться до домноження оригіналу на $-t$:
		\begin{equation}
			\mathcal{L}^{-1} \left\{ F'(s)\right\} = -t \cdot f(t)
		\end{equation}
		або, взагалі:
		\begin{equation}
			\mathcal{L}^{-1} \left\{ F^{(n)}(s)\right\} = (-t)^{n} \cdot f(t)
		\end{equation}
	\item Інтегрування оригіналу. Зводиться до поділу зображення Лапласа на $s$:
		\begin{equation}
			\label{eq:laplace_int_origin}
			\Lapl{\int\limits_0^t f(\tau) d\tau} = \frac{F(s)}{s}
		\end{equation}
	\item Інтегрування зображення. Якщо інтеграл
	$$\int\limits_s^\infty F(p) dp$$
	збігається, то він є зображенням для функції $\frac{f(t)}{t}$:
		\begin{equation}
			\label{eq:laplace_int_image}
			\Lapl{\frac{f(t)}{t}} = \int\limits_s^\infty F(p) dp
		\end{equation}
	\item Теорема зміщення. Для $\forall s_0 \in \CC$:
		\begin{equation}
			\Lapl{e^{s_0 t}f(t)} = F(s - s_0)
		\end{equation}
	\item Теорема запізнення. Для $\forall t_0 > 0$
		\begin{equation}
			\Lapl{f(t - t_0)} = e^{-s t_0} F(s)
		\end{equation}
		або більш загально, $\forall t \in \RR$
		\begin{equation}
			\Lapl{f(t - t_0) \eta(t - t_0)} = e^{-s t_0} F(s)
			\label{eq:laplace_lag}
		\end{equation}
		де $\eta(t) = 
		\begin{cases}
			0, &t < 0\\
			1, &t ≥ 0
		\end{cases}
		\text{– функція Хевісайда.}
		$

\hspace*{\dimexpr\linewidth-\textwidth\relax} Важливою для застосування є наступна теорема:

		\item Теорема єдиності. Якщо дві функції $f_1(t)$ та $f_1(t)$ мають одне й те саме зображення Лапласа $F(s)$, то ці функції тотожно рівні.
\end{enumerate}
\begin{defin}
Згорткою двох функцій $f(t)$ та $g(t)$ називається
	\begin{equation}
		(f*g)(t) = \int\limits_{-\infty}^{\infty} f(\tau) g(t - \tau) d\tau
	\end{equation}
\end{defin}
\begin{enumerate}[label=\arabic*.]
	\setcounter{enumi}{9}
	\item Теорема про згортку. Зображенням згортки двох функцій є добуток зображень Лапласа цих функцій:
		\begin{equation}
			\Lapl{(f*g)(t)} = F(s) \cdot G(s)
		\end{equation}
\end{enumerate}
Для визначення зв’язку між асимптотичною поведінкою оригіналу та зображення було використано \hyperref[eq:tauber_thm]{теорему Таубера}.
