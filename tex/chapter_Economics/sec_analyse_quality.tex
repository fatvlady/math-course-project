\section{Аналіз рівня якості варіантів реалізації функцій}
Визначаємо рівень якості кожного варіанту виконання основних функцій окремо. 

Абсолютні значення параметрів Х2(об’єм пам’яті для збереження даних) та X1 (швидкодія мови програмування) відповідають технічним вимогам умов функціонування даного ПП. 

Абсолютне значення параметра Х3 (час обробки зображення) обрано не найгіршим (не максимальним), тобто це значення відповідає або варіанту а) 800 мс або варіанту б) 80мс. 

Коефіцієнт технічного рівня для кожного варіанта реалізації ПП розраховується за формулою~(\ref{eq:economics:quality_of_functions}) Результати навередемо в таблиці~\ref{tab:economics:quality_of_functions}
\begin{equation}
	\label{eq:economics:quality_of_functions}
	K_K(j) = \sum\limits_{i=1}^{n} \weightcoef{i,j} B^{(i,j)}
\end{equation}
де $ n $ - кількість параметрів, $ \weightcoef{i} $ - коефіцієнт вагомості $i$-го параметра, $ B^{(i,j)} $ - оцінка $i$-го параметра в балах.

\begin{table}
	\caption{Розрахунок показників рівня якості варіантів реалізації основних функцій ПП}
	\centering
	\begin{tabularx}{1.0\textwidth}{|X|X|X|X|X|X|}
		\hline
		Основні функції & Варіант реалізації функції & Абсолютне значення параметра & 
			Бальна оцінка параметра & Коефіцієнт вагомості параметра & Коефіцієнт рівня якості \\
		\hline
		$F_1(X_1)$ & А & 11000 	& 3.6 	& 0.215 & 0.774	\\
		\hline
		$F_2(X_3)$ & А & 800	& 2.4 	& 0.348 & 0.835	\\
		\hline
		\multirow{2}{*}{$F_3(X_2,X_4)$} 
			   & А &16& 3.4& 0.283 &0.962 \\
		\cline{2-6}
			   & Б & 80 	& 1 	& 0.154 & 0.154 \\
		\hline
	\end{tabularx}
	
	\label{tab:economics:quality_of_functions}
\end{table}

\newcommand{\coef}[2]{K_\text{#1}\left[ #2 \right] }
За даними з таблиці~\ref{tab:economics:quality_of_functions} за формулою
\begin{equation}
	\label{eq:economics:var_quality}
	K_K = \coef{ТУ}{F_{1k}} + \coef{ТУ}{F_{2k}} + \ldots + \coef{ТУ}{F_{zk}} 
\end{equation}
 визначаємо рівень якості кожного з варіантів:
 \begin{equation*}
	 \begin{array}{ccc}
	 K_{K1} =& 0.774 + 0.835 + 0.962 &= 2.57 \\
	 K_{K2} =& 0.774 + 0.835 + 0.154 &= 1.76 
	 \end{array}
 \end{equation*}
Як видно з розрахунків, кращим є перший варіант, для якого коефіцієнт технічного рівня має найбільше значення. 