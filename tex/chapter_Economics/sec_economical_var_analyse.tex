\section{Економічний аналіз варіантів розробки ПП}

Для визначення вартості розробки ПП спочатку проведемо розрахунок трудомісткості. 

Всі варіанти включають в себе два окремих завдання:
\begin{enumerate}
	\item \label{it:economics:exe1}Розробка проекту програмного продукту;
	\item \label{it:economics:exe2}Розробка програмної оболонки; 
\end{enumerate}

Завдання \ref{it:economics:exe1} за ступенем новизни відноситься до групи А, завдання \ref{it:economics:exe2} – до групи Б. За складністю алгоритми, які використовуються в завданні \ref{it:economics:exe1} належать до групи 1; а в завданні \ref{it:economics:exe2} – до групи 3. 

Для реалізації завдання \ref{it:economics:exe1} використовується довідкова інформація, а завдання \ref{it:economics:exe2}  використовує інформацію у вигляді даних. 

Проведемо розрахунок норм часу на розробку та програмування для кожного з завдань. Загальна трудомісткість обчислюється як 
\newcommand{\coe}[2]{\text{#1}_\text{#2}}
\begin{equation}
	\label{eq:economics:laboriousness}
	\coe{Т}{О} = \coe{Т}{Р} 
	\cdot
	\coe{К}{П}
	\cdot
	\coe{К}{СК}
	\cdot
	\coe{К}{М}
	\cdot
	\coe{К}{СТ}
	\cdot
	\coe{К}{СТ.М}
\end{equation}

де $ \coe{Т}{Р} $~-~трудомісткість розробки ПП; $ \coe{К}{П} $~-~поправочний коефіцієнт; $ \coe{К}{СК} $~-~коефіцієнт на складність вхідної інформації; $ \coe{К}{М} $~-~коефіцієнт рівня мови програмування; $ \coe{К}{СТ} $~-~коефіцієнт використання стандартних модулів і прикладних програм; $ \coe{К}{СТ.М} $~-~коефіцієнт стандартного математичного забезпечення.

Для першого завдання, виходячи із норм часу для завдань розрахункового характеру степеню новизни А та групи складності алгоритму 1, трудомісткість дорівнює: $ \coe{Т}{Р} = 90 $ людино-днів. Поправний коефіцієнт, який враховує вид нормативно-довідкової інформації для першого завдання: $ \coe{К}{П} = 1.7 $. Поправний коефіцієнт, який враховує складність контролю вхідної та вихідної інформації для всіх завдань рівний $1$: $ \coe{К}{СК} = 1 $. Оскільки при розробці першого завдання використовуються стандартні модулі, врахуємо це за допомогою коефіцієнта $ \coe{К}{СТ} = 0.8 $. Тоді, за формулою~(\ref{eq:economics:laboriousness}), загальна трудомісткість програмування першого завдання дорівнює: 
\begin{equation*}
	T_1 = 90\cdot 1.7\cdot 0.8 = 122.4 \quad\text{людино-днів}
\end{equation*}

Проведемо аналогічні розрахунки для подальших завдань. 

Для другого завдання (використовується алгоритм третьої групи складності, степінь новизни Б), тобто $ \coe{Т}{Р} =27 $ людино-днів, $ \coe{К}{П} = 0.9 $, $ \coe{К}{СК} = 1 $, $ \coe{К}{СТ} =0.8 $:
\begin{equation*}
T_2 = 27\cdot 0.9\cdot 0.8 = 19.44\quad\text{людино-днів}
\end{equation*}

Складаємо трудомісткість відповідних завдань для кожного з обраних варіантів реалізації програми, щоб отримати їх трудомісткість, враховуючи більшу складність другого завдання в другому випадку: 
\begin{align*}
T_1 = & (122.4 + 19.44 + 4.8) \cdot 8 	&= 1173.12 \text{ людино-годин} \\
T_2 = & (122.4 + 19.44 + 6.91) \cdot 8 	&= 1190 \text{ людино-годин} 
\end{align*}

Найбільш високу трудомісткість має варіант 2. В розробці беруть участь два програмісти з окладом 6000~грн., один аналітик-програміст з окладом 9000~грн. Визначимо годинну зарплату за формулою: 
\begin{equation}
	C_h = \cfrac{M}{T_m t} \text{ грн}
\end{equation}
де $ M $ – середній місячний оклад працівників; $ T_m $ – кількість робочих днів на місяць; $t$ – кількість робочих годин в день. 

\begin{equation}
	C_h = \cfrac{ 6000 + 6000 + 9000}{3 \cdot 21 \cdot 8}  = 41.67 \text{ грн}
\end{equation}

Тоді, розрахуємо заробітну плату за формулою:

\begin{equation}
	\coe{C}{зп} = C_h \cdot T_i \cdot \coe{К}{КД}
\end{equation}
де $ C_h $– величина погодинної оплати праці програміста; $ T_i $ – трудомісткість відповідного завдання; $ \coe{К}{КД} $ – норматив, який враховує додаткову заробітну плату. 

Зарплата розробників за варіантами становить: 
\begin{enumerate}
	\item $ \coe{C}{зп} = 41.67\cdot 1328.64 \cdot 1.2 = 58660.69  \text{ грн}$
	\item $ \coe{C}{зп} = 41.67\cdot 1345.52 \cdot 1.2 = 59504.76  \text{ грн} $
\end{enumerate}

Відрахування на єдиний соціальний внесок в залежності від групи професійного ризику (II клас) становить 22\%: 
\begin{enumerate}
	\item $ \coe{C}{від} = \coe{C}{зп} \cdot 0.22 = 58660.69\cdot 0.22 = 12905.35  \text{ грн}$
	\item $ \coe{C}{від} = \coe{C}{зп} \cdot 0.22 = 59504.76 \cdot 0.22 = 13091.05  \text{ грн}  $
\end{enumerate}

Тепер визначимо витрати на оплату однієї машино-години — $ \coe{C}{М} $

Так як одна ЕОМ обслуговує одного програміста з окладом 6000~грн., з коефіцієнтом зайнятості $ 0.2 $ то для однієї машини отримаємо: 
\begin{equation*}
\coe{C}{Г} = 12 \cdot M \cdot \coe{К}{З} = 12 \cdot 6000\cdot 0.2 = 14400 \text{ грн} 
\end{equation*}

З урахуванням додаткової заробітної плати: 
$$\coe{C}{зп} =\coe{С}{г}\cdot (1+ KЗ) = 14400\cdot (1 + 0.2)=17280  \text{ грн}$$

Відрахування на єдиний соціальний внесок:

$$\coe{C}{від}= \coe{C}{зп} \cdot 0.22 = 17280\cdot 0.22 = 3801.6  \text{ грн}$$

Амортизаційні відрахування розраховуємо при амортизації 20\% та вартості ЕОМ – 8000 грн.

$$\coe{C}{а} = \coe{К}{тм}\cdot \coe{K}{а} \cdot \coe{Ц}{пр} = 1.15 \cdot 0.2 \cdot 8000 = 1840  \text{ грн}$$

де $ \coe{К}{тм} $ – коефіцієнт, який враховує витрати на транспортування та монтаж приладу у користувача; $ \coe{K}{а} $ – річна норма амортизації; $ \coe{Ц}{пр} $ – договірна ціна приладу.

Витрати на ремонт та профілактику розраховуємо як:
$$
\coe{С}{р} = 
	\coe{К}{тм} \cdot \coe{Ц}{пр} \cdot \coe{К}{р} = 
	1.15 \cdot 8000 \cdot 0.05 = 460  \text{ грн}
$$
де $ \coe{К}{р} $– відсоток витрат на поточні ремонти.

Ефективний годинний фонд часу ПК за рік розраховуємо за формулою:
\begin{multline*}
\coe{Т}{еф}
	= (\coe{Д}{к} - \coe{Д}{в} - \coe{Д}{с} - \coe{Д}{р}) \cdot tЗ\cdot \coe{К}{в} = \\
	= (365 - 104 - 8 - 16) \cdot 8 \cdot 0.9 = 1706.4 \text{ годин}
\end{multline*}

де $ \coe{Д}{к} $ – календарна кількість днів у році; $ \coe{Д}{в}, \coe{Д}{с} $ – відповідно кількість вихідних та святкових днів; $ \coe{Д}{р} $ – кількість днів планових ремонтів устаткування; $ t $ – кількість робочих годин в день; $ \coe{К}{в} $– коефіцієнт використання приладу у часі протягом зміни.

Витрати на оплату електроенергії розраховуємо за формулою:
$$\coe{С}{ел} = \coe{Т}{еф}\cdot N_c\cdot \coe{К}{з}\cdot \coe{Ц}{ен} =1706.4 \cdot 0.156 \cdot 0.2436 \cdot 2.08 = 134.87  \text{ грн} $$ %TODO

де $ N_c $ – середньо-споживча потужність приладу; $ \coe{К}{з} $ – коефіцієнтом зайнятості приладу; $ \coe{Ц}{ен} $ – тариф за 1 КВт-годину електроенергії.

Накладні витрати розраховуємо за формулою:
$$\coe{С}{н} = \coe{Ц}{пр} \cdot 0.67 = 8000\cdot 0.67 = 5360  \text{ грн}$$

Тоді, річні експлуатаційні витрати будуть:
\begin{multline*}
\coe{С}{екс} = 
\coe{C}{зп}+ \coe{C}{від}+ \coe{C}{а} + \coe{С}{р}+ \coe{С}{ел} + \coe{С}{н} = \\
	  = 17280 + 3801.6 + 1840 + 460 + 134.87 + 5360 = 28876.47 \text{ грн}
\end{multline*}
Собівартість однієї машино-години ЕОМ дорівнюватиме:
$$
	\coe{С}{м-г} = \coe{С}{екс}/ \coe{Т}{еф} = 28876.47/1706.4 = 16.92 \text{ грн/год}
$$
В нашому випадку всі роботи пов’язані з розробкою програмного продукту ведуться на ЕОМ. Витрати на оплату машинного часу розраховуються за наступною формулою:
$$
	\coe{С}{м} = \coe{С}{м-г}  \cdot T
$$
В залежності від обраного варіанта реалізації, витрати на оплату машинного часу складають:
\begin{enumerate}
	\item $ \coe{С}{м} = 16.92\cdot 1173.12 = 19849.19  \text{ грн} $	
	\item  $ \coe{С}{м} = 16.92\cdot 1190 = 20134.8  \text{ грн} $
\end{enumerate}

Накладні витрати складають 67\% від заробітної плати:
$$\coe{С}{н} = \coe{C}{зп} \cdot 0.67$$
\begin{enumerate}
\item $ \coe{С}{н} = 58660.69\cdot 0.67 = 39302.66  \text{ грн} $
\item $ \coe{С}{н} = 59504.76\cdot 0.67 = 39868.19  \text{ грн} $
\end{enumerate}

Отже, вартість розробки ПП за варіантами становить:
$$
	\coe{C}{пп} = \coe{C}{зп}+ \coe{C}{від}+ \coe{С}{м} +\coe{С}{н}
$$
\begin{enumerate}
	\item $ \coe{C}{пп} = 58660.69 + 12905.35 + 19849.19 + 39302.66 = 130717.89  \text{ грн} $
	\item $ \coe{C}{пп} = 59504.76 + 13091.05 + 20134.8 + 39868.19 = 132598.8 \text{ грн}$
\end{enumerate}