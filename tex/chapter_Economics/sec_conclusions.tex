\section{Висновки до розділу}
В даному розділі проведено повний функціонально-вартісний аналіз ПП, який було розроблено в рамках дипломного проекту. Процес аналізу можна умовно розділити на дві частини.

В першій з них проведено дослідження ПП з технічної точки зору: було визначено основні функції ПП та сформовано множину варіантів їх реалізації; на основі обчислених значень параметрів, а також експертних оцінок їх важливості було обчислено коефіцієнт технічного рівня, який і дав змогу визначити оптимальну з технічної точки зору альтернативу реалізації функцій ПП.

Другу частину ФВА присвячено вибору з альтернативних варіантів реалізації найбільш економічно обґрунтованого. Порівняння запропонованих варіантів реалізації в рамках даної частини виконувалось за коефіцієнтом ефективності, для обчислення якого були обчислені такі допоміжні параметри, як трудомісткість, витрати на заробітну плату, накладні витрати.

Після виконання функціонально-вартісного аналізу програмного комплексу що розроблюється, можна зробити висновок, що з альтернатив, що залишились після першого відбору двох варіантів виконання програмного комплексу оптимальним є перший варіант реалізації програмного продукту. У нього виявився найкращий показник техніко-економічного рівня якості:
$$ 
	\coe{K}{тер} = 0.197\cdot 10^{-4}
$$
Цей варіант реалізації програмного продукту має такі параметри:
\begin{itemize}
	\item мова програмування – С++ 
	\item розрахунки — бібліотека STL
	\item інтерфейс користувача — консольного типу
\end{itemize}
Даний варіант виконання програмного продукту надає користувачу зручний інтерфейс консольного типу, непоганий функціонал і гнучкість з достатньо великою швидкодією.