\section{Висновки до розділу}
У даному розділі було проведено асимптотичний аналіз поведінки математичного сподівання максимальної кількості автомобілів на парковці при достатньо великих лінійних розмірах парковки.

Було розглянуто 3 тривіальних моделі паркування автомобілів, 2 з яких є крайовими, тобто визначають верхню та нижню межу кількості автомобілів на парковці.

Було розглянуто 2 нетривіальні моделі паркування автомобілів. Для них виведено аналітичні формули асимптотичної поведінки математичного сподівання максимальної кількості автомобілів на парковці при достатньо великих розмірах парковки:
\begin{enumerate}
\item Водії розподіляються рівномірно по вільному проміжку на парковці. Отримано результат \eqref{eq:uniform_base_final}.
\item Узагальнення першої моделі. Водії з ймовірністю $p$ ставлять свій автомобіль скраю вільного проміжку, а з ймовірністю $q = 1 - p$ – аналогічно першій нетривіальній моделі, – рівномірно. Отримано результат \eqref{eq:uniform_right_final}.
\item Покращено асимптотичну оцінку для узагальнення моделі. Отримано результат \eqref{eq:uniform_right_as_enhanced}.
\end{enumerate}

Аналітично отримані константи знаходяться чисельно, що буде виконано у наступному розділі.