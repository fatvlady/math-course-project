\section{Аналіз архітектури продукту}

Продукт "modeler" побудований таким чином, що він зчитує параметри заданою користувачем моделі у форматі xml, та в циклі моделювати траекторію процесу паркування автомобілів на одновимірній парковці, використовуючи вбудований в C++ генератор псевдовипадкових чисел. За результат програма бере середнє значення по всім змодельованим траекторіям.

Продукт "modeler2d" побудований таким чином, що він зчитує параметри заданою користувачем моделі у форматі xml, та в циклі моделювати траекторію процесу паркування автомобілів на двовимірній парковці, використовуючи вбудований в C++ генератор псевдовипадкових чисел. За результат програма бере середнє значення по всім змодельованим траекторіям. Ідея багаторазової ітерації з подальшим взяттям середнього ґрунтується на законі великих чисел.

Продукт "integral" вираховує подвійні інтеграли, наведені у розділі \ref{chapter:theory}.
