\section{Керівництво користувача}
\jointitles
%Опишемо керівництво користувача для двох частин програмного продукту.

\subsection{Основний програмний продукт}

Програмний продукт розроблено як додаток консольного типу. Тобто, для оперування роботою додатка використовуються текстові команди, що вводяться в консоль операційної системи.

Запуск програми "modeler" відбувається за допомогою команди:
\begin{verbatim}
	modeler [<model_name>.xml]
\end{verbatim}

Запуска програми "modeler2d" відбувається аналогічно:

\begin{verbatim}
	modeler2d [<model_name>.xml]
\end{verbatim}

\subsubsection{Основні параметри}

Користувач задає необхідну для вивчення модель через xml-файл. Типова структура xml-файлу виглядає наступним чином:
\begin{verbatim}
<model repeat_count="<count>" parking_length="<length>">
  <behaviour_1 p="<prob>"/>
  ...
  <behaviour_n p="<prob>"/>
</model>
\end{verbatim}

Тобто в моделі задаються варіанти поведінки водіїв, а також відповідні ймовірності вибору поведінки. Істотним зауваженням є те, що сума ймовірностей має дорівнювати 1, інакше програма завершує роботу з ненульовим кодом.

Параметри моделі для продукту "modeler" перелічені в таблиці~\ref{tab:modeler_model_params}.

\begin{table}[ht]
	\caption{Параметри моделі для продукту "modeler"}
	\centering
\begin{tabular}{|p{0.3\textwidth}|p{0.65\textwidth}|}
	\hline
	Параметр & Пояснення \\
	\hline
	\verb!repeat_count! & Кількість запусків моделювань для вирахування середнього значення \\
	\hline
	\verb!parking_length! & Довжина парковки для моделювання \\
	\hline
\end{tabular}	
	\label{tab:modeler_model_params}
\end{table}

Типи дисциплін водіїв, що підтримуються продуктом "modeler" перелічені в таблиці~\ref{tab:modeler_behaviours}.

\begin{table}[ht]
	\caption{Дисципліни водіїв, що підтримуються продуктом "modeler"}
	\centering
\begin{tabular}{|p{0.3\textwidth}|p{0.65\textwidth}|}
	\hline
	Дисципліна & Пояснення \\
	\hline
	\verb!left! & Водій ставить свій автомобіль зліва вільного проміжку \\
	\hline
	\verb!right! & Водій ставить свій автомобіль справа вільного проміжку \\
	\hline
	\verb!center! & Водій ставить свій автомобіль по центру вільного проміжку \\
	\hline
	\verb!uniform! & Водій ставить свій автомобіль керуючись рівномірним розподілом \\
	\hline
\end{tabular}	
	\label{tab:modeler_behaviours}
\end{table}

Для продукту "modeler2d" було підключено лише поведінку водіїв, аналогічну класичній моделі Реньї у випадку одновимірної моделі, тобто за рівномірним розподілом. Тому для цієї програми задаються лише параметри моделі, перелічені в таблиці~\ref{tab:modeler2d_model_params}.

\begin{table}[ht]
	\caption{Параметри моделі для продукту "modeler2d"}
	\centering
\begin{tabular}{|p{0.3\textwidth}|p{0.65\textwidth}|}
	\hline
	Параметр & Пояснення \\
	\hline
	\verb!repeat_count! & Кількість запусків моделювань для вирахування середнього значення \\
	\hline
	\verb!a! & Довжина парковки по вісі $0x$ \\
	\hline
	\verb!b! & Довжина парковки по вісі $0y$ \\
	\hline
\end{tabular}	
	\label{tab:modeler2d_model_params}
\end{table}



