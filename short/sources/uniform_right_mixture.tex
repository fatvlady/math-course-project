\section{Дослідження моделі з вибором місця для авто за сумішшю рівномірного та розподілу Бернулі}
\label{sec:uniform_right}
%TODO:

В цій моделі водії вибирають місце для автомобіля, керуючись наступним правилом
\begin{itemize}
\item з ймовірністю $\alpha$ водій ставить автомобіль в правому кінці вільного проміжку, 
\item з ймовірністю $\beta = 1 - \alpha$ водій вибирає місце керуючись рівномірним розподілом, аналогічно тому, як це робилося у главі \ref{sec:uniform_base}.
\end{itemize}

\subsection{Виведення інтегрального рівняння}

Аналогічно, як і в попередній частині, порядок вибору вільних проміжків водіями не впливає на результат, тому будемо вважати, що після паркування одного автомобіля парковка розбивається на 2 частини, і після цього спочатку заповнюється ліва частина, а потім права.

Необхідно визначити $m(X)=\EE F(X)$. Нехай $\xi \sim Uniform(0, X - 1)$ – випадкова величина, що визначає положення лівого краю першого автомобіля на парковці у випадку вибору місця за рівномірним розподілом. Тоді маємо наступну тотожність:
\[
\begin{split}
	m(X) = \alpha(1+ m(X-1)) + \beta(1+ \EE(\EE(F(\xi) + F(X - 1 - \xi)  | \xi))) =\\
	=1 + \alpha m(X-1) + \beta \int\limits_0^{X-1} m(t) \frac{1}{X − 1} dt + \beta \int\limits_0^{X-1} m(X − t − 1) \frac{1}{X − 1} dt 
\end{split}
\]

\begin{equation}
\phi(\overline{u}) = \int_0^\infty M(x) e^{-\overline{u}x} dx = \int_0^\infty M(x) \overline{e^{-sx}} dx = \overline{\phi(u)}
\end{equation}

Аналогічно виведенню формули \eqref{eq:uniform_equation}, отримаємо

\begin{equation}
	\label{eq:uniform_right_equation}
	m(X + 1) = 1 + \alpha m(X) + \frac{2\beta}{X} \int\limits_0^{X} m(t) dt,\quad \forall X > 0
\end{equation}

Аналогічно діям в главі \ref{sec:uniform_base}, маємо, що виконується \eqref{eq:zero_state}, \eqref{eq:bounds} та \eqref{eq:bounds_inference}.

\subsection{Перехід до зображення Лапласа}
\label{sec:uniform_right_integral_laplace}

Спробуємо розв'язати \eqref{eq:uniform_right_equation} за допомогою перетворення Лапласа.

Оскільки виконується \eqref{eq:bounds}, то зображення Лапласа для $m(X)$ існує. До того ж, аналогічно до виведення в главі \ref{sec:uniform_base} маємо, що зображення Лапласа існує і для інших доданків в правій частині рівняння \eqref{eq:uniform_right_equation}.

Таким чином, отримали інтегральне рівняння в термінах зображення Лапласа, яке вже можна розв'язати, адже нема зсуву:
\begin{equation}
	\label{eq:uniform_right_laplace_integral}
	e^s M(s) = \alpha M(s) + 2\beta \int\limits_s^\infty \frac{M(u)}{u} du + \frac{1}{s}
\end{equation}

Продиференціюємо обидві частини рівняння за $s$:
\begin{equation}
	e^s M(s) + e^s \dot M(s) =\alpha \dot M(s) - 2 \beta \frac{M(s)}{s}  - \frac1{s^2}
\end{equation}

Виразимо $\dot M(s)$ з цього рівняння:
\begin{equation}
	\label{eq:uniform_right_laplace_diff}
	\dot M(s) = - M(s)\left(\frac{e^s}{e^s - \alpha} + \frac{2\beta}{s(e^s - \alpha)}\right) - \frac{1}{s^2(e^s - \alpha)}
\end{equation}

Розв'яжемо отримане диференційне рівняння. Спочатку розв'яжемо однорідну частину:
\begin{align*}
	\dot M_h (s) &= - M_h (s)\left(\frac{e^s}{e^s - \alpha} + \frac{2\beta}{s(e^s - \alpha)}\right) \\
	\frac{\dot M_h (s)}{M_h (s)} &= -\left(\frac{e^s}{e^s - \alpha} + \frac{2\beta}{s(e^s - \alpha)}\right) \\
	\int\limits_1^s \frac{\dot M_h (u)}{M_h (u)} du &= -\int\limits_1^s \left(\frac{e^u}{e^u - \alpha} + \frac{2\beta}{u(e^u - \alpha)}\right) du \\
	\left. \ln{M_h (u)}\right|_1^s &= - \int\limits_1^s \frac{e^u}{e^u - \alpha} du  - 2 \int\limits_1^s \frac{\beta}{u(e^u - \alpha)} du
\end{align*}

Позначимо
\begin{equation}
	\label{eq:almost_li_alpha}
	Q_\alpha(s) := \int\limits_1^s \frac{\beta}{u(e^u - \alpha)} du
\end{equation}

Оскільки
\begin{align*}
	\int\limits_1^s \frac{e^u}{e^u - \alpha} du &= \< u = e^u - \alpha,~ du = e^u du = (u+\alpha) du \> = \\
	&= \int\limits_{e-\alpha}^{e^s -\alpha} \frac{u + \alpha}{u} (u+\alpha)^{-1} du = 
	\int\limits_{e-\alpha}^{e^s -\alpha} \frac{du}{u} = \\
	&= \log(e^s - \alpha) - \log(e-\alpha) = \log \frac{e^s - \alpha}{e - \alpha} \text{,}
\end{align*}

то
\begin{align*}
	\ln{M_h (s)} &= \ln{M_h (1)} - \log \frac{e^s - \alpha}{e - \alpha} - 2Q_\alpha(s) \\
	M_h (s) &= M_h (1) \cdot \frac{e - \alpha}{e^s - \alpha} \cdot e^{- 2Q_\alpha(s)} \cdot const
\end{align*}

Оскільки $M(1)$ та $(e-\alpha)$ можна включити в константу, то маємо розв'язок
\begin{equation}
	\label{eq:uniform_right_laplace_homogen_sol}
	M_h (s) = C \cdot \left((e^s - \alpha)  e^{2Q_\alpha(s)}\right)^{-1}, \quad \forall C \in \RR
\end{equation}

Дійсно, перевіримо цей розв'язок:
\[
\begin{split}
	\dot M_h (s) &= C \cdot \left(\frac{1}{(e^s - \alpha)  e^{2Q_\alpha(s)}}\right)'=\\
	 &= - C \cdot \left(\frac{1}{(e^s - \alpha)  e^{2Q_\alpha(s)}}\right)^2 \cdot \left(e^s  + (e^s - \alpha) \frac{2\beta}{s(e^s - \alpha)}   \right) e^{2Q_\alpha(s)} =\\
	 &= -C \left((e^s - \alpha)  e^{2Q_\alpha(s)}\right)^{-1} \cdot \left(\frac{e^s}{e^s - \alpha}  + \frac{2\beta}{s(e^s - \alpha)}   \right) =\\
	 &= - M(s) \cdot \left(\frac{e^s}{e^s - \alpha}  + \frac{2\beta}{s(e^s - \alpha)}   \right)
\end{split}
\]

Нескладно помітити, що отримали вихідне рівняння. Тепер застосуємо метод варіації довільних сталих:
$$
	M(s) = C(s) \cdot \left((e^s - \alpha)  e^{2Q_\alpha(s)}\right)^{-1} 
$$

Продиференціювавши за $s$, отримаємо:
\begin{align*}
	\dot M(s) &= \dot C(s) \cdot \left((e^s - \alpha)  e^{2Q_\alpha(s)}\right)^{-1} -C(s) \left((e^s - \alpha)  e^{2Q_\alpha(s)}\right)^{-1} \cdot \\
	&\cdot \left(\frac{e^s}{e^s - \alpha}  + \frac{2\beta}{s(e^s - \alpha)}   \right) 
\end{align*}

З іншої сторони, з \eqref{eq:uniform_right_laplace_diff} маємо
$$
	 \dot M(s) = - C(s) \left((e^s - \alpha)  e^{2Q_\alpha(s)}\right)^{-1}  \left(\frac{e^s}{e^s - \alpha} + \frac{2\beta}{s(e^s - \alpha)}\right) - \frac{1}{s^2(e^s - \alpha)} \quad	
$$
Тому
$$
	\dot C(s) = - \left((e^s - \alpha)  e^{2Q_\alpha(s)}\right) \cdot  \frac{1}{s^2(e^s - \alpha)} =  - \frac{e^{2Q_\alpha(s)}}{s^2}
$$

Тоді простим інтегруванням в межах від 1 до $s$ отримуємо:
\begin{equation}
	C(s) = -  \int\limits_1^s \frac{e^{2Q_\alpha(u)}}{u^2} du + const
\end{equation}

І тоді отримуємо вираз для $M(s)$:
\begin{equation}
\begin{split}
	M(s)=- \left( \int\limits_1^s \frac{e^{2Q_\alpha(u)}}{u^2} du + const \right) \left((e^s - \alpha)  e^{2Q_\alpha(s)}\right)^{-1}  =\\
	= - \left( \int\limits_1^s \frac{e^{2Q_\alpha(u)}}{u^2} du + K \right) \frac{1}{(e^s - \alpha)  e^{2Q_\alpha(s)}}, \quad K \in \RR
\end{split}
\end{equation}

Перевіримо отриманий результат:
\begin{equation*}
\begin{split}
	&\dot M(s) = -\left( \int\limits_1^s \frac{e^{2Q_\alpha(u)}}{u^2} du + K \right)' \frac{1}{(e^s - \alpha)  e^{2Q_\alpha(s)}} -\left( \int\limits_1^s \frac{e^{2Q_\alpha(u)}}{u^2} du + K \right)\cdot \\ 
	&\cdot \left(\frac{1}{(e^s - \alpha)  e^{2Q_\alpha(s)}}\right)' = -\frac{e^{2Q_\alpha(s)}}{s^2} \frac{1}{(e^s - \alpha)  e^{2Q_\alpha(s)}}  + \left( \int\limits_1^s \frac{e^{2Q_\alpha(u)}}{u^2} du + K \right) \cdot \\
	& \cdot \left((e^s - \alpha)  e^{2Q_\alpha(s)}\right)^{-1} \cdot \left(\frac{e^s}{e^s - \alpha}  + \frac{2\beta}{s(e^s - \alpha)}   \right)= -\frac{1}{s^2(e^s - \alpha)} -\\
	&- M(s)\left(\frac{e^s}{e^s - \alpha}  + \frac{2\beta}{s(e^s - \alpha)}   \right)
\end{split}
\end{equation*}

Перевірено. Тоді остаточний результат без вирахування константи:
\begin{equation}
	\label{eq:uniform_right_laplace_sol_initial}
	M_\alpha(s)= \left( \int\limits_s^1 \frac{e^{2Q_\alpha(u)}}{u^2} du + K \right) \frac{1}{(e^s - \alpha)  e^{2Q_\alpha(s)}}
\end{equation}

Зазначимо, що у випадку $\alpha=0$ ми маємо випадок з глави \ref{sec:uniform_base}, тож і формула \eqref{eq:uniform_right_laplace_sol_initial} має співпадати з \eqref{eq:uniform_laplace_sol_initial} при $\alpha = 0$. Спираючись на те, що $Q_0(s) \equiv Q(s)$, отримуємо, що $M_0(s) \equiv M(s) \; \forall s > 0$.

\subsection{Визначення константи у розв'язку}

У \ref{sec:uniform_integral_laplace} було доведено, що зображення Лапласа існує не тільки для $m(X)$, а і для $m(X+1)$, до того ж,
\begin{equation}
	\Lapl{m(X+1)} = e^s M(s).
\end{equation}

Аналогічно доводиться той самий факт, але для випадку суміші рівномірного розподілу та розподілу Бернулі. Таким чином, маємо
\begin{equation}
	\Lapl{m_\alpha(X+1)} =\tilde M_\alpha(s)= \left( \int\limits_s^1 \frac{e^{2Q_\alpha(u)}}{u^2} du + K \right) \frac{e^s}{e^s - \alpha} e^{- 2Q_\alpha(s)}.
\end{equation}

Оскільки зображення Лапласа – аналітична функція в деякій правій півплощині комплексного простору, то $\tilde M_\alpha(s) \rightarrow 0,\; s \rightarrow +\infty$.

Розглянемо $Q_\alpha(s)$ ($s$ розглядаємо на дійсній вісі):
\begin{equation}
\begin{split}
	\label{eq:q_alpha_s_major}
	&Q_\alpha(s) = \int\limits_1^s \frac{1-\alpha}{u(e^u - \alpha)} du <  \int\limits_1^\infty \frac{1 - \alpha}{u(e^u - \alpha)} du < 
	\int\limits_1^\infty \frac{1- \alpha}{e^u - \alpha} du<\\
	&< \int\limits_1^\infty \frac{1}{e^u} du = \exp(-1) - \exp(-\infty) = \exp(-1)
\end{split}
\end{equation}

Останній перехід нерівності пояснюється досить просто:
$$
\frac{1-\alpha}{u - \alpha} < \frac{1}{\alpha}, \; u > 1 \Leftrightarrow u - u \alpha = u(1-\alpha) < u - \alpha, \; u > 1
$$

Тобто $Q_\alpha(s)$ - обмежена на $[1; \infty]$. Тому обмеженими на цій вісі будуть і $e^{\pm 2Q_\alpha(s)}$. Також зрозуміло, що якщо інтегрувати по дійсній вісі, то $Q_\alpha(s)$ – монотонно зростаюча за $s$. Тому
\begin{equation}
	0 = \tilde M_\alpha(\infty) = \lim_{s\rightarrow \infty} \tilde M_\alpha(s) = \left( \int\limits_\infty^1 \frac{e^{2Q_\alpha(u)}}{u^2} du + K \right) \lim_{s\rightarrow \infty} e^{- 2Q_\alpha(s)}
\end{equation}

Тут $\lim\limits_{s\rightarrow \infty} e^{- 2Q_\alpha(s)} = const > 0$, тому маємо, що
\begin{equation}
	K = -  \int\limits_\infty^1 \frac{e^{2Q_\alpha(u)}}{u^2} du =  \int\limits_1^\infty \frac{e^{2Q_\alpha(u)}}{u^2} du.
\end{equation}

Таким чином, отримали нову версію $M(s)$:
\begin{equation}
\begin{split}
	\label{eq:uniform_right_laplace_sol}
	M_\alpha(s)&= \left( \int\limits_s^1 \frac{e^{2Q_\alpha(u)}}{u^2} du + K \right) \frac{1}{(e^s - \alpha)  e^{2Q_\alpha(s)}}
 = \\
	&=\frac{1}{(e^s - \alpha)  e^{2Q_\alpha(s)}} \int\limits_s^\infty \frac{e^{2Q_\alpha(u)}}{u^2} du
\end{split}
\end{equation}

\subsection{Застосування теореми Таубера}

Для знаходження асимптотики $m_\alpha(X)$ на нескінченності, за теоремою Таубера \eqref{eq:tauber_thm} необхідно визначити асимптотику $M_\alpha(s)$ при $s \rightarrow 0$.

Якщо знайти такі $C \in \RR$ та $\delta \in \RR^+$, що $M_\alpha(s) \sim C \cdot s^{-\delta}, \; s \rightarrow 0$, то можна стверджувати, що $\int\limits_0^X m_\alpha(x) dx \sim \frac{1}{\Gamma(\delta + 1)} C X^\delta, \; X \rightarrow \infty$. Вже зараз зрозуміло, що $\delta = 2$, адже теорема справедлива в обидва боки і виконується \eqref{eq:bounds_inference}.

Для цього розглянему поведінку в нулі трьох множників, з яких складається $M_\alpha(s)$, а саме:
\begin{enumerate}
	\item $\frac{1}{e^s - \alpha}$;
	\item $e^{-2Q_\alpha(s)}$;
	\item $\int\limits_s^\infty \frac{e^{2Q_\alpha(u)}}{u^2} du$,
\end{enumerate}

Щодо першого множнику, то в 0 він, очевидно, прямує до $\frac{1}{1-\alpha}$,

Для наступного аналізу доведемо деякі леми.

\begin{lem}
	\label{eq:exp_q_alpha_s_asymptotics}
	$e^{-2Q(s)}$ поводиться як $s^{-2}$ в 0, з точністю до константи, а саме:
	\begin{equation}
		\lim\limits_{s \rightarrow 0} \frac{e^{-2Q(s)}}{s^{-2}} = \exp\left(-2\int\limits_0^1 \frac{e^u  - 1}{u(u^u - \alpha)} du\right)
	\end{equation}
\end{lem}
\begin{proof}
	Для знаходження ліміту прологарифмуємо вираз. Отримаємо:
	\[
	\begin{split}
	2 \ln s - 2 Q_\alpha(s) = 2 \ln s &- 2 \int\limits_1^s \frac{1 - \alpha}{u(e^u - \alpha)} du = 2 \int\limits_1^s \frac{1}{u} du - \\
	- 2 \int\limits_1^s \frac{1 - \alpha}{u(e^u - \alpha)} du &= 2  \int\limits_1^s \frac{e^u - 1}{u(e^u - \alpha)} du = -2 \int\limits_s^1 \frac{e^u - 1}{u(e^u - \alpha)} du
	\end{split}
	\]
	Тепер, підвівши до експоненти обидві частини, отримаємо:
	$$
		\frac{e^{-2Q_\alpha(s)}}{s^{-2}} = \exp\left(-2 \int\limits_s^1 \frac{e^u - 1}{u(e^u - \alpha)} du\right)
	$$
	Якщо довести, що інтеграл
	$$
		-2 \int\limits_0^1 \frac{e^u - 1}{u(e^u - \alpha)} du
	$$
	збігається, то лему буде доведено, адже експонента – неперервна функція, і можна переходити до ліміту під експонентою.
	Зрозуміло, що
	$$
		-2 \int\limits_s^1 \frac{e^u - 1}{u(e^u - \alpha)} du
	$$
	збігається для $\forall s \in (0;~1]$. Дійсно, оскільки $e^u - 1 < e^u - \alpha$, підінтегральна функція $ \frac{1 - e^{-u}}{u}$ мажорується $\frac{1}{u}$, яка, в свою чергу, має скінченне значення інтегралу:
	$$
		\int\limits_s^1 \frac{1}{u} du = \ln 1 - \ln s = -\ln s,\quad s > 0
	$$
	Невизначеність виникає лише в точці 0. Знайдемо ліміт підінтегральної функції в точці 0:
\begin{align*}
	\lim\limits_{s\rightarrow 0} \frac{e^u - 1}{u(e^u - \alpha)} = \<\text{правило Лопіталя для невизначенності 0/0}\> = \\
	= \lim\limits_{s\rightarrow 0} \frac{e^u}{u e^u + (e^u-\alpha)} = \frac{1}{1-\alpha}
\end{align*}
	Таким чином, підінтегральна функція обмежена в деякому $\varepsilon$-околі 0, тому інтеграл також збіжний, і лему доведено.
\end{proof}

\begin{lem}
	\label{eq:q_alpha_s_limited}
	Функція
	$$
		Q_\alpha(s) = \int\limits_1^s \frac{1- \alpha}{u(e^u - \alpha)} du
	$$
	  – обмежена на $[w; ~\infty], \; w>0$.
\end{lem}
\begin{proof}
	На проміжку $[1; ~\infty]$ підінтегральна функція мажорується функцією $e^{-u}$ (див. \eqref{eq:q_alpha_s_major}), а на проміжку $[w; ~1]$ – функцією $\frac{1}{u}$, адже $1- \alpha < e^u - \alpha, ~ u > 0$. Тому, аналогічно доведенню попередньо леми, інтеграл буде збіжний, і:
	\begin{align*}
		Q_\alpha(s) \leq \int\limits_1^\infty e^{-u} du = \exp(-1) &,\quad s \geq 1 \\
		Q_\alpha(s) \leq \int\limits_w^1 \frac{1}{u} du = - \ln w &, \quad s \in [w;~1]
	\end{align*}
	Таким чином, $Q_\alpha(s) \leq \max\{-\ln w, \exp(-1)\}$.
\end{proof}

\begin{lem}
	Інтеграл
	$$
		\int\limits_0^\infty \frac{e^{2Q_\alpha(u)}}{u^2} du
	$$
	– збіжний.
\end{lem}
\begin{proof}
	Спираючись на лему \eqref{eq:exp_q_alpha_s_asymptotics}, маємо, що підінтегральна функція прямує до деякої константи при $u \rightarrow 0$, оскільки є обернено пропорційною до функції з тої леми. Тому в деякому проколотому $\varepsilon$-околі точки 0 підінтегральна функція буде обмежена. На інтервалі $[\varepsilon; ~\infty]$ за лемою \eqref{eq:q_alpha_s_limited}, $Q_\alpha(u)$ – обмежена, а тому і $\exp(2Q_\alpha(u))$ також. Тому збіжність на інтервалі $[\varepsilon; ~\infty]$ виконується, якщо збігається інтеграл
	\[
		\int\limits_\varepsilon^\infty \frac{1}{u^2} du.
	\]
	А його збіжність – відомий факт.
\end{proof}
		
Таким чином, спираючись на доведені леми, маємо при $s \rightarrow 0$:
\begin{align*}
&M_\alpha(s) \sim \frac{s^{-2}}{1-\alpha} \cdot \exp\left(-2\int\limits_0^1 \frac{e^u  - 1}{u(e^u - \alpha)} du\right) \int\limits_0^\infty \frac{e^{2Q_\alpha(u)}}{u^2} du = \\
& = \frac{s^{-2}}{1-\alpha} \cdot \exp\left(-2\int\limits_0^1 \frac{e^u  - 1}{u(e^u - \alpha)} du\right) \int\limits_0^\infty \exp\left(2\int\limits_1^u \frac{1- \alpha}{\tau(e^\tau - \alpha)} d\tau - 2 \ln u\right) du = \\
& = \frac{s^{-2}}{1-\alpha} \cdot \exp\left(-2\int\limits_0^1 \frac{e^u  - 1}{u(e^u - \alpha)} du\right) \int\limits_0^\infty \exp\left(2\int\limits_1^u \frac{1- \alpha}{\tau(e^\tau - \alpha)} d\tau - 2  \int\limits_1^u  \frac{1}{\tau} d\tau \right) du = \\
& = \frac{s^{-2}}{1-\alpha} \cdot \exp\left(-2\int\limits_0^1 \frac{e^u  - 1}{u(e^u - \alpha)} du\right) \int\limits_0^\infty \exp\left(-2\int\limits_1^u \frac{e^{\tau} - 1}{\tau(e^\tau - \alpha)} d\tau  \right) du = \\
& = \frac{s^{-2}}{1-\alpha} \cdot \int\limits_0^\infty \exp\left( -2\int\limits_0^1 \frac{e^{\tau} - 1}{\tau(e^\tau - \alpha)} d\tau -2\int\limits_1^u \frac{e^{\tau} - 1}{\tau(e^\tau - \alpha)} d\tau  \right) du 
\end{align*}

Склавши інтеграли під експонентою, отримаємо:
\begin{equation}
	M_\alpha(s) \sim s^{-2} \cdot \frac{1}{1-\alpha} \int\limits_0^\infty \exp\left( -2\int\limits_0^u \frac{e^{\tau} - 1}{\tau(e^\tau - \alpha)} d\tau  \right) du, \quad s \rightarrow 0.
\end{equation}

Тепер, за теоремою Таубера маємо при $X \rightarrow \infty$:
\begin{equation}
	\int\limits_0^X m_\alpha(x) dx \sim \frac{1}{(1-\alpha)\Gamma(2 + 1)} \int\limits_0^\infty \exp\left( -2\int\limits_0^u \frac{e^{\tau} - 1}{\tau(e^\tau - \alpha)} d\tau  \right) du \cdot X^2.
\end{equation}

Або, продиференціювавши обидві частини,  отримаємо:
\begin{equation}
	m_\alpha(X) \sim \frac{2}{(1- \alpha) \Gamma(2 + 1)} \int\limits_0^\infty \exp\left( -2\int\limits_0^u \frac{e^{\tau} - 1}{\tau(e^\tau - \alpha)} d\tau  \right) du \cdot X.
\end{equation}
\begin{equation}
	\label{eq:uniform_right_final}
	m_\alpha(X) \sim \frac{1}{1-\alpha} \int\limits_0^\infty \exp\left( -2\int\limits_0^u \frac{e^{\tau} - 1}{\tau(e^\tau - \alpha)} d\tau  \right) du \cdot X, \quad X \rightarrow \infty.
\end{equation}

Тут ми мали право диференціювати обидві частини за правилом Лопіталя, адже має місце невизначеність $\infty / \infty$.

