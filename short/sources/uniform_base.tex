\section{Дослідження моделі з вибором місця для авто за рівномірним розподілом}
%TODO:
\label{sec:uniform_base}

В цій моделі водії вибирають місце для автомобіля, керуючись рівномірним розподілом.

\subsection{Виведення інтегрального рівняння}

Зрозуміло, що порядок вибору вільних проміжків водіями не впливає на результат, тому будемо вважати, що після паркування одного автомобіля парковка розбивається на 2 частини, і після цього спочатку заповнюється ліва частина, а потім права.

Необхідно визначити $m(X)=\EE F(X)$. Нехай $\xi \sim Uniform(0, X - 1)$ – випадкова величина, що визначає положення лівого краю першого автомобіля на парковці. Тоді маємо наступну тотожність:
\[
\begin{split}
	m(X) = \EE(\EE(F(\xi) + F(X - 1 - \xi) + 1 | \xi)) =\\
	=\int\limits_0^{X-1} m(t) \frac{1}{X − 1} dt + \int\limits_0^{X-1} m(X − t − 1) \frac{1}{X − 1} dt + 1
\end{split}
\]

Так як
$$\int\limits_0^{X-1} m(X − t − 1) dt = \< u=X - t - 1 \> = - \int\limits_{X-1}^0 m(u) dt = \int\limits_0^{X-1} m(u)dt,$$

то
\begin{equation}
	m(X) = \frac{2}{X-1} \int\limits_0^{X-1} m(t) dt + 1
\end{equation}

Для зручності зробимо заміну $X \rightarrow X + 1$. Отримаємо

\begin{equation}
	\label{eq:uniform_equation}
	m(X + 1) = \frac{2}{X} \int\limits_0^{X} m(t) dt + 1,\quad \forall X > 0
\end{equation}

Таким чином, отримали інтегральне рівняння. До того ж, відомо, що
\begin{equation}
	\label{eq:zero_state}
	m(X) \equiv 0, \quad X \in [0; 1)
\end{equation}

Спираючись на \eqref{eq:upperbound} та \eqref{eq:lowerbound}, маємо обмеження на $m(X)$:
\begin{equation}
	\label{eq:bounds}
	\left[\frac{X+1}{2}\right] \leq m(X) \leq \left[X\right]
\end{equation}

З цієї нерівності випливає, що якщо є якась асимптотика у функції $m(X)$, то вона порядку $X$, тобто
\begin{equation}
	\label{eq:bounds_inference}
	m(X) \sim C\cdot X \text{ при } x \rightarrow +\infty, \quad C \in [0.5; 1]
\end{equation}

\subsection{Перехід до зображення Лапласа}
\label{sec:uniform_integral_laplace}

Спробуємо розв'язати \eqref{eq:uniform_equation} за допомогою перетворення Лапласа.

Оскільки виконується \eqref{eq:bounds}, то зображення Лапласа для $m(X)$ існує. До того ж, за властивістю \eqref{eq:laplace_lag}:
$$\Lapl{m(X+1)}=\<\eqref{eq:zero_state}\>=\Lapl{m(X+1)\eta(X+1)}=e^s M(s).$$

Оскільки $m(X) \leq X$, то $\int\limits_0^X m(t) dt < X^2$, тобто для інтегралу від $m(X)$ зображення також існує, за властивістю \eqref{eq:laplace_int_origin}:
$$\Lapl{\int\limits_0^X m(t) dt} = \frac{M(s)}{s}.$$

Аналогічно доводиться, що $\frac{1}{X}\int\limits_0^X m(t) dt < X$ при $X>0$, а тому зображення Лапласа для цього виразу також існує. Тоді за властивістю \eqref{eq:laplace_int_image}:
$$\Lapl{\frac{2}{X}\int\limits_0^X m(t) dt} = 2 \int\limits_s^\infty \frac{M(u)}{u} du.$$

Таким чином, отримали інтегральне рівняння в термінах зображення Лапласа, яке вже можна розв'язати, адже нема зсуву:
\begin{equation}
	\label{eq:uniform_laplace_integral}
	e^s M(s) = 2 \int\limits_s^\infty \frac{M(u)}{u} du + \frac{1}{s}
\end{equation}

Продиференціюємо обидві частини рівняння за $s$:
\begin{equation}
	e^s M(s) + e^s \dot M(s) = - 2\frac{M(s)}{s}  - \frac1{s^2}
\end{equation}

Виразимо $\dot M(s)$ з цього рівняння:
\begin{equation}
	\label{eq:uniform_laplace_diff}
	\dot M(s) = -M(s) - \frac{2M(s)}{se^s} - \frac{1}{s^2 e^s} = - M(s)\left(1 + \frac{2e^{-s}}{s}\right) - \frac{e^{-s}}{s^2}
\end{equation}

Розв'яжемо отримане диференційне рівняння. Спочатку розв'яжемо однорідну частину:
\begin{align*}
	\dot M_h (s) &= - M_h (s)\left(1 + \frac{2e^{-s}}{s}\right) \\
	\frac{\dot M_h (s)}{M_h (s)} &= -\left(1 + \frac{2e^{-s}}{s}\right) \\
	\int\limits_1^s \frac{\dot M_h (u)}{M_h (u)} du &= -\int\limits_1^s \left(1 + \frac{2e^{-u}}{u}\right) du \\
	\left. \ln{M_h (u)}\right|_1^s &= - (s - 1) - 2 \int\limits_1^s \frac{e^{-u}}{u} du
\end{align*}

Позначимо
\begin{equation}
	\label{eq:almost_li}
	Q(s) := \int\limits_1^s \frac{e^{-u}}{u} du
\end{equation}

Тоді маємо
\begin{align*}
	\ln{M_h (s)} &= \ln{M_h (1)} + 1 - s - 2Q(s) \\
	M_h (s) &= M_h (1) \cdot e^{1 - s - 2Q(s)} \cdot const
\end{align*}

Оскільки $M(1)$ можна включити в константу, то маємо розв'язок
\begin{equation}
	\label{eq:uniform_laplace_homogen_sol}
	M_h (s) = C \cdot e^{1 - s - 2Q(s)}, \quad \forall C \in \RR
\end{equation}

Дійсно, перевіримо цей розв'язок:
\[
\begin{split}
	&\dot M_h (s) = C \cdot e^{1 - s - 2Q(s)} \cdot (-1 - 2 \dot Q(s)) =\\
	 &=C \cdot e^{1 - s - 2Q(s)} \cdot \left(-1 - 2\frac{e^{-s}}{s}\right)=
	- M_h (s) \cdot \left(1 + 2\frac{e^{-s}}{s}\right)
\end{split}
\]

Нескладно помітити, що отримали вихідне рівняння. Тепер застосуємо метод варіації довільних сталих:
\begin{align*}
	M(s) &= C(s) \cdot e^{1 - s - 2Q(s)} \\
	\dot M(s) &= \dot C(s) \cdot e^{1 - s - 2Q(s)} - C(s) \cdot (e^{1 - s - 2Q(s)}) \cdot \left(1 + 2 \frac{e^{-s}}{s}\right) \\
	&\text{з іншої сторони, з \eqref{eq:uniform_laplace_diff} маємо} \\
	 \dot M(s) &= - C(s) \cdot e^{1 - s - 2Q(s)} \left(1 + 2 \frac{e^{-s}}{s}\right) - \frac{e^{-s}}{s^2} \quad	\\
	 &\text{тому} \\
	 \dot C(s) &= -\frac{e^{-s}}{s^2} e^{s+2Q(s) - 1} = -e^{-1}\frac{e^{2Q(s)}}{s^2} \\
\end{align*}

Тоді простим інтегруванням в межах від 1 до $s$ отримуємо:
\begin{equation}
	C(s) = - \exp(-1) \cdot \int\limits_1^s \frac{e^{2Q(u)}}{u^2} du + const
\end{equation}

І тоді отримуємо вираз для $M(s)$:
\begin{equation}
\begin{split}
	M(s)=- \left( \exp(-1) \cdot \int\limits_1^s \frac{e^{2Q(u)}}{u^2} du + const \right) e^{1 - s - 2Q(s)} =\\
	= - \left( \int\limits_1^s \frac{e^{2Q(u)}}{u^2} du + K \right) e^{- s - 2Q(s)}, \quad K \in \RR
\end{split}
\end{equation}

Перевіримо отриманий результат:
\begin{equation*}
\begin{split}
	&\dot M(s) = -\left( \int\limits_1^s \frac{e^{2Q(u)}}{u^2} du + K \right)' e^{- s - 2Q(s)} -\left( \int\limits_1^s \frac{e^{2Q(u)}}{u^2} du + K \right)\cdot \\ 
	&\cdot \left(e^{- s - 2Q(s)}\right)' = -\frac{e^{2Q(s)}}{s^2} e^{- s - 2Q(s)}  + \left( \int\limits_1^s \frac{e^{2Q(u)}}{u^2} du + K \right) \cdot \\
	& \cdot e^{- s - 2Q(s)} \left(1 + 2 \frac{e^{-s}}{s}\right) = -\frac{e^{-s}}{s^2} - M(s)\left(1 + 2 \frac{e^{-s}}{s}\right)
\end{split}
\end{equation*}

Перевірено. Тоді остаточний результат без вирахування константи:
\begin{equation}
	\label{eq:uniform_laplace_sol_initial}
	M(s)= \left( \int\limits_s^1 \frac{e^{2Q(u)}}{u^2} du + K \right) e^{- s - 2Q(s)}
\end{equation}

\subsection{Визначення константи у розв'язку}

У \ref{sec:uniform_integral_laplace} було доведено, що зображення Лапласа існує не тільки для $m(X)$, а і для $m(X+1)$, до того ж,
\begin{equation}
	\Lapl{m(X+1)} = e^s M(s).
\end{equation}

Таким чином, маємо
\begin{equation}
	\Lapl{m(X+1)} =\tilde M(s)= \left( \int\limits_s^1 \frac{e^{2Q(u)}}{u^2} du + K \right) e^{- 2Q(s)}.
\end{equation}

Оскільки зображення Лапласа – аналітична функція в деякій правій півплощині комплексного простору, то $\tilde M(s) \rightarrow 0,\; s \rightarrow +\infty$.

Розглянемо $Q(s)$ ($s$ розглядаємо на дійсній вісі):
\begin{equation}
\begin{split}
	&Q(s) = \int\limits_1^s \frac{e^{-u}}{u} du <  \int\limits_1^\infty \frac{e^{-u}}{u} du < 
	\int\limits_1^\infty e^{-u} du=\\
	&= \exp(-1) - \exp(-\infty) = \exp(-1)
\end{split}
\end{equation}

Тобто $Q(s)$ - обмежена на $[1; \infty]$. Тому обмеженими на цій вісі будуть і $e^{\pm 2Q(s)}$. Також зрозуміло, що якщо інтегрувати по дійсній вісі, то $Q(s)$ – монотонно зростаюча за $s$. Тому
\begin{equation}
	0 = \tilde M(\infty) = \lim_{s\rightarrow \infty} \tilde M(s) = \left( \int\limits_\infty^1 \frac{e^{2Q(u)}}{u^2} du + K \right) \lim_{s\rightarrow \infty} e^{- 2Q(s)}
\end{equation}

Тут $\lim\limits_{s\rightarrow \infty} e^{- 2Q(s)} = const > 0$, тому маємо, що
\begin{equation}
	K = -  \int\limits_\infty^1 \frac{e^{2Q(u)}}{u^2} du =  \int\limits_1^\infty \frac{e^{2Q(u)}}{u^2} du.
\end{equation}

Таким чином, отримали нову версію $M(s)$:
\begin{equation}
	\label{eq:uniform_laplace_sol}
	M(s)= \left( \int\limits_s^1 \frac{e^{2Q(u)}}{u^2} du + K \right) e^{- s - 2Q(s)} = 
	e^{- s - 2Q(s)} \int\limits_s^\infty \frac{e^{2Q(u)}}{u^2} du
\end{equation}

\subsection{Застосування теореми Таубера}

Для знаходження асимптотики $m(X)$ на нескінченності, за теоремою Таубера \eqref{eq:tauber_thm} необхідно визначити асимптотику $M(s)$ при $s \rightarrow 0$.

Якщо знайти такі $C \in \RR$ та $\alpha \in \RR^+$, що $M(s) \sim C \cdot s^{-\alpha}, \; s \rightarrow 0$, то можна стверджувати, що $\int\limits_0^X m(x) dx \sim \frac{1}{\Gamma(\alpha + 1)} C X^\alpha, \; X \rightarrow \infty$. Вже зараз зрозуміло, що $\alpha = 2$, адже теорема справедлива в обидва боки і виконується \eqref{eq:bounds_inference}.

Для цього розглянему поведінку в нулі трьох множників, з яких складається $M(s)$, а саме:
\begin{enumerate}
	\item $e^{-s}$;
	\item $e^{-2Q(s)}$;
	\item $\int\limits_s^\infty \frac{e^{2Q(u)}}{u^2} du$,
\end{enumerate}

Щодо першого множнику, то в 0 він прямує до 1, тож ні на що не впливає, і його можна не розглядати.

Для наступного аналізу доведемо деякі леми.

\begin{lem}
	\label{eq:exp_q_s_asymptotics}
	$e^{-2Q(s)}$ поводиться як $s^{-2}$ в 0, з точністю до константи, а саме:
	\begin{equation}
		\lim\limits_{s \rightarrow 0} \frac{e^{-2Q(s)}}{s^{-2}} = \exp\left(-2\int\limits_0^1 \frac{1-e^{-u}}{u} du\right)
	\end{equation}
\end{lem}
\begin{proof}
	Для знаходження ліміту прологарифмуємо вираз. Отримаємо:
	\[
	\begin{split}
	2 \ln s - 2 Q(s) = 2 \ln s - 2 \int\limits_1^s \frac{e^{-u}}{u} du &= 2 \int\limits_1^s \frac{1}{u} du - 2 \int\limits_1^s \frac{e^{-u}}{u} du = \\
		= 2  \int\limits_1^s \frac{1 - e^{-u}}{u} du &= -2 \int\limits_s^1 \frac{1 - e^{-u}}{u} du
	\end{split}
	\]
	Тепер, підвівши до експоненти обидві частини, отримаємо:
	$$
		\frac{e^{-2Q(s)}}{s^{-2}} = \exp\left(-2 \int\limits_s^1 \frac{1 - e^{-u}}{u} du\right)
	$$
	Якщо довести, що інтеграл
	$$
		-2 \int\limits_0^1 \frac{1 - e^{-u}}{u} du
	$$
	збігається, то лему буде доведено, адже експонента – неперервна функція, і можна переходити до ліміту під експонентою.
	Зрозуміло, що
	$$
		-2 \int\limits_s^1 \frac{1 - e^{-u}}{u} du
	$$
	збігається для $\forall s \in (0;~1]$. Дійсно, підінтегральна функція $ \frac{1 - e^{-u}}{u}$ мажорується $\frac{1}{u}$, яка, в свою чергу, має скінченне значення інтегралу:
	$$
		\int\limits_s^1 \frac{1}{u} du = \ln 1 - \ln s = -\ln s,\quad s > 0
	$$
	Невизначеність виникає лише в точці 0. Знайдемо ліміт підінтегральної функції в точці 0:
	$$
		\lim\limits_{s\rightarrow 0} \frac{1 - e^{-u}}{u} = \<\text{правило Лопіталя для невизначенності 0/0}\> = \lim\limits_{s\rightarrow 0} \frac{e^{-u}}{1} = 1
	$$
	Таким чином, підінтегральна функція обмежена в деякому $\varepsilon$-околі 0, тому інтеграл також збіжний, і лему доведено.
\end{proof}

\begin{lem}
	\label{eq:q_s_limited}
	Функція
	$$
		Q(s) = \int\limits_1^s \frac{e^{-u}}{u} du
	$$
	  – обмежена на $[w; ~\infty], \; w>0$.
\end{lem}
\begin{proof}
	На проміжку $[1; ~\infty]$ підінтегральна функція мажорується функцією $e^{-u}$, а на проміжку $[w; ~1]$ – функцією $\frac{1}{u}$, тому, аналогічно доведенню попередньої леми, інтеграл буде збіжний, і:
	\begin{align*}
		Q(s) \leq \int\limits_1^\infty e^{-u} du = \exp(-1) &,\quad s \geq 1 \\
		Q(s) \leq \int\limits_w^1 \frac{1}{u} du = - \ln w &, \quad s \in [w;~1]
	\end{align*}
	Таким чином, $Q(s) \leq \max\{-\ln w, \exp(-1)\}$.
\end{proof}

\begin{lem}
	Інтеграл
	$$
		\int\limits_0^\infty \frac{e^{2Q(u)}}{u^2} du
	$$
	– збіжний.
\end{lem}
\begin{proof}
	Спираючись на лему \eqref{eq:exp_q_s_asymptotics}, маємо, що підінтегральна функція прямує до деякої константи при $u \rightarrow 0$, оскільки є обернено пропорційною до функції з тої леми. Тому в деякому проколотому $\varepsilon$-околі точки 0 підінтегральна функція буде обмежена. На інтервалі $[\varepsilon; ~\infty]$ за лемою \eqref{eq:q_s_limited}, $Q(u)$ – обмежена, а тому і $\exp(2Q(u))$ також. Тому збіжність на інтервалі $[\varepsilon; ~\infty]$ виконується, якщо збігається інтеграл
	\[
		\int\limits_\varepsilon^\infty \frac{1}{u^2} du.
	\]
	А його збіжність – відомий факт.
\end{proof}
		
Таким чином, спираючись на доведені леми, маємо:
\begin{align*}
&M(s) \sim s^{-2} \cdot \exp\left(-2\int\limits_0^1 \frac{1-e^{-u}}{u} du\right) \int\limits_0^\infty \frac{e^{2Q(u)}}{u^2} du = \\
& = s^{-2} \cdot \exp\left(-2\int\limits_0^1 \frac{1-e^{-u}}{u} du\right) \int\limits_0^\infty \exp\left(2\int\limits_1^u \frac{e^{-\tau}}{\tau} d\tau - 2 \ln u\right) du = \\
& = s^{-2} \cdot \exp\left(-2\int\limits_0^1 \frac{1-e^{-u}}{u} du\right) \int\limits_0^\infty \exp\left(2\int\limits_1^u \frac{e^{-\tau}}{\tau} d\tau - 2  \int\limits_1^u  \frac{1}{\tau} d\tau \right) du = \\
& = s^{-2} \cdot \exp\left(-2\int\limits_0^1 \frac{1-e^{-u}}{u} du\right) \int\limits_0^\infty \exp\left(-2\int\limits_1^u \frac{1 - e^{-\tau}}{\tau} d\tau  \right) du = \\
& = s^{-2} \cdot \int\limits_0^\infty \exp\left( -2\int\limits_0^1 \frac{1-e^{-\tau}}{\tau} du -2\int\limits_1^u \frac{1 - e^{-\tau}}{\tau} d\tau  \right) du 
\end{align*}

Склавши інтеграли під експонентою, отримаємо:
\begin{equation}
	M(s) \sim s^{-2} \cdot \int\limits_0^\infty \exp\left( -2\int\limits_0^u \frac{1 - e^{-\tau}}{\tau} d\tau  \right) du, \quad s \rightarrow 0.
\end{equation}

Тепер, за теоремою Таубера маємо при $X \rightarrow \infty$:
\begin{equation}
	\int\limits_0^X m(x) dx \sim \frac{1}{\Gamma(2 + 1)} \int\limits_0^\infty \exp\left( -2\int\limits_0^u \frac{1 - e^{-\tau}}{\tau} d\tau  \right) du \cdot X^2.
\end{equation}

Або, продиференціювавши обидві частини,  отримаємо:
\begin{equation}
	m(X) \sim \frac{2}{\Gamma(2 + 1)} \int\limits_0^\infty \exp\left( -2\int\limits_0^u \frac{1 - e^{-\tau}}{\tau} d\tau  \right) du \cdot X.
\end{equation}
\begin{equation}
	\label{eq:uniform_base_final}
	m(X) \sim  \int\limits_0^\infty \exp\left( -2\int\limits_0^u \frac{1 - e^{-\tau}}{\tau} d\tau  \right) du \cdot X, \quad X \rightarrow \infty.
\end{equation}

Тут ми мали право диференціювати обидві частини за правилом Лопіталя, адже має місце невизначеність $\infty / \infty$.

