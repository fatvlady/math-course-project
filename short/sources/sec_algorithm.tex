\section{Аналіз роботи алгоритму}
\jointitles

\subsection{Алгоритм моделювання на одновимірній парковці}

Для моделювання одновимірної парковки виконується наступний алгоритм дій:
\begin{enumerate}
\item Створюється вектор довжин вільних проміжків, який ініціалізується однією довжиною – довжиною парковки. 
\item Допоки вектор не пустий, виконується: 
\subitem Вибирається поведінка водія використовуючи рандомізатор.
\subitem Дістається остання довжина з вектору, і ділиться на частини відповідно до моделі поведінки водія.
\subitem Кожна з частин додається до вектору, якщо її довжина не менша 1, тобто, довжини автомобіля.
\subitem Інкрементується поточна кількість автомобілів.
\end{enumerate}

Цей процес повторюється задану кількість разів, і в кінці програма видає середнє значення автомобілів з усіх ітерацій, а також відношення до довжини парковки.

\subsection{Алгоритм моделювання на двовимірній парковці}

Двовимірний алгоритм працює аналогічно одновимірному, тільки розглядаються не проміжки, а прямокутники. А саме, створюється список вільних прямокутників. Під вільним розуміється прямокутник, в якому будь-яка точка може бути потенційним центром автомобіля.

Ініціалізується цей список початковим прямокутником – $(\frac{1}{2}, \frac{1}{2}, a - \frac{1}{2}, b - \frac{1}{2})$, де $a,~b$ – задані лінійні розміри парковки. Спираючись на площі поточних прямокутників, обирається один з ймовірністю рівною відношенню його площі до сумарної площі усіх вільних прямокутників. Після цього всередині обраного прямокутника обирається за рівномірним розподілом точка, навколо якої будується квадратний окіл розміру 2x2, і перевіряється на перетин з усіма поточними вільними прямокутниками. Якщо перетинається, то вільний прямокутник ділиться на менші прямокутники, при чому один із прямокутників – область перетину. Усі менші прямокутники, окрім перетину, додаються в список.
